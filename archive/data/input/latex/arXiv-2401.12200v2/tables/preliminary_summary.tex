% Please add the following required packages to your document preamble:
% \usepackage{booktabs}
\begin{table*}[t!]
\centering
\small
\begin{tabular}{@{}ll|cc|cc|cc@{}}
\toprule
\multicolumn{2}{l|}{\multirow{2}{*}{Method}} & \multirow{2}{*}{$\mathcal{A}_{\text{P}}$} & \multirow{2}{*}{$\mathcal{A}_{\text{T}}$} & \multicolumn{2}{c|}{Training}                                             & \multicolumn{2}{c}{Inference}                                                        \\
\multicolumn{2}{l|}{}                        &                     &                     & T                                          & M                         & T                                                  & M                          \\ \midrule
\multirow{3}{*}{PEFT}       & Adapter~\citep{Pfeiffer2020AdapterFusionNT}        & \xmark              & \xmark              & \color{red}$\Uparrow_{\text{High}}$ & \color{ForestGreen}$\Downarrow_{\text{Low}}$ & \color{red}$\Uparrow_{\text{Low}}$                              & \color{red}$\Uparrow_{\text{Low}}$      \\
                            & LoRA~\citep{hu_lora_2021}           & \xmark              & \xmark              & \color{red}$\Uparrow_{\text{High}}$ & \color{ForestGreen}$\Downarrow_{\text{Low}}$ & \textbf{=}                                         & \textbf{=}                 \\
                            & AdaLoRA~\citep{zhang2023adaptive}        & \xmark              & \cmark              & \color{red}$\Uparrow_{\text{High}}$ & \color{ForestGreen}$\Downarrow_{\text{Low}}$ & \textbf{=}                                         & \textbf{=}                 \\ \midrule
\multirow{4}{*}{Pruning}    & MvP~\citep{sanh_movement_2020}            & \xmark              & \xmark              & \color{red}$\Uparrow_{\text{High}}$ & \color{red}$\Uparrow_{\text{Low}}$     & \color{ForestGreen}$\Downarrow_{\text{Low}}$                         & \color{ForestGreen}$\Downarrow_{\text{Low}}$ \\
                            & BMP~\citep{lagunas_block_2021}            & \xmark              & \xmark              & \color{red}$\Uparrow_{\text{High}}$ & \color{red}$\Uparrow_{\text{Low}}$     & \color{ForestGreen}$\Downarrow_{\text{High}}$ & \color{ForestGreen}$\Downarrow_{\text{Low}}$  \\
                            & CoFi~\citep{xia_structured_2022}           & \xmark              & \xmark              & \color{red}$\Uparrow_{\text{High}}$ & \color{red}$\Uparrow_{\text{Low}}$     & \color{ForestGreen}$\Downarrow_{\text{High}}$ & \color{ForestGreen}$\Downarrow_{\text{Low}}$  \\
                            & MT~\citep{kwon_fast_2022}             & \xmark              & \xmark              & \textbf{=}                                 & \textbf{=}                & \color{ForestGreen}$\Downarrow_{\text{High}}$ & \color{ForestGreen}$\Downarrow_{\text{Low}}$  \\ \midrule
\multirow{3}{*}{Combined}   & SPA~\citep{hedegaard_structured_2022}            & \xmark              & \xmark              & \color{red}$\Uparrow_{\text{High}}$ & \color{red}$\Uparrow_{\text{Low}}$     & \color{ForestGreen}$\Downarrow_{\text{High}}$ & \color{ForestGreen}$\Downarrow_{\text{Low}}$  \\
                            & LRP~\citep{zhang2023pruning}            & \xmark              & \xmark              & \color{red}$\Uparrow_{\text{High}}$ & \color{ForestGreen}$\Downarrow_{\text{Low}}$ & \color{ForestGreen}$\Downarrow_{\text{High}}$ & \color{ForestGreen}$\Downarrow_{\text{Low}}$  \\
                            & \textbf{\ourmethod} (ours)   & \cmark              & \cmark              & \color{red}$\Uparrow_{\text{Low}}$                      & \color{ForestGreen}$\Downarrow_{\text{Low}}$ & \color{ForestGreen}$\Downarrow_{\text{High}}$ & \color{ForestGreen}$\Downarrow_{\text{Low}}$  \\ \bottomrule
\end{tabular}
\caption{Efficiency comparison of existing methods and APT. $\mathcal{A}_{\text{P}}$ stands for adaptive pruning and $\mathcal{A}_{\text{T}}$ for adaptive tuning, where the total and tuning parameter sizes are dynamically adjusted. We measure efficiency using training converge time, inference time (T), and peak memory (M). Symbols {\color{red}$\Uparrow$} and {\color{ForestGreen}$\Downarrow$} indicate increased and decreased costs, respectively, while \textbf{=} signifies no change in cost. The terms ``low'' and ``high'' qualify the extent of cost variations.} 
\label{tab:preliminary-summary}
\vspace{-15pt}
\end{table*}