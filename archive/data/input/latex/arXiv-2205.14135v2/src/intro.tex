\section{Introduction}
\label{sec:intro}





Transformer models~\citep{vaswani2017attention} have emerged as the most widely used architecture in applications such as natural language processing and image classification.
Transformers have grown larger~\citep{brown2020language} and deeper~\citep{wang2022deepnet}, but equipping them with longer context remains difficult~\citep{tay2020long}, since the self-attention module at their heart has time and memory complexity quadratic in sequence length.
An important question is whether making attention faster and more memory-efficient can help Transformer models address their runtime and memory challenges for long sequences.

Many approximate attention methods have aimed to reduce the compute and memory requirements of attention.
These methods range from sparse-approximation~\citep{kitaev2020reformer, roy2021efficient} to low-rank approximation~\citep{wang2020linformer, katharopoulos2020transformers, choromanski2020rethinking}, and their combinations~\citep{beltagy2020longformer, zaheer2020bigbird, scatterbrain}.
Although these methods reduce the compute requirements to linear or near-linear in sequence length, many of them do not display wall-clock speedup against standard attention and have not gained wide adoption. One main reason is that they focus on FLOP reduction (which may not correlate with wall-clock speed) and tend to ignore overheads from memory access (IO).





\iftoggle{icmlworkshop}{
\begin{figure*}[t]
}{
\begin{figure}[t]
}
\centering
\includegraphics[width=5.5in]{figs/banner_pdf.pdf}
\caption{
\textbf{Left:} \sysname uses tiling to prevent materialization of the large $N \times N$ attention matrix (dotted box) on (relatively) slow GPU HBM. In the outer loop (red arrows), \sysname loops through blocks of the $\vK$ and $\vV$ matrices and loads them to fast on-chip SRAM.
In each block, \sysname loops over blocks of $\vQ$ matrix (blue arrows), loading them to SRAM, and writing the output of the attention computation back to HBM.
\textbf{Right:} Speedup over the PyTorch implementation of attention on GPT-2.
\sysname does not read and write the large $N\times N$ attention matrix to HBM, resulting in an 7.6$\times$ speedup on the attention computation.}
\label{fig:banner}
\iftoggle{arxiv}{}{
\vspace{-2em}
}
\iftoggle{icmlworkshop}{
\end{figure*}
}{
\end{figure}
}






In this paper, we argue that a missing principle is making attention algorithms \textit{IO-aware}~\citep{aggarwal1988input}---that is, carefully accounting for reads and writes to different levels of fast and slow memory (e.g., between fast GPU on-chip SRAM and relatively slow GPU high bandwidth memory, or HBM~\citep{jia2018dissecting}, Figure~\ref{fig:banner} left).
On modern GPUs, compute speed has out-paced memory speed~\citep{nvidia2017nvidia,nvidia2020nvidia,nvidia2022nvidia}, and most operations in Transformers are bottlenecked by memory accesses~\citep{ivanov2021data}.
IO-aware algorithms have been critical for similar memory-bound operations, when reading and writing data can account for a large portion of the runtime---such as database joins~\citep{ramakrishnan2003database}, image processing~\citep{ragan2013halide}, numerical linear algebra~\citep{blackford2002updated}, and more~\citep{williams2009roofline, hennessy2003memory}.
However, common Python interfaces to deep learning such as PyTorch and Tensorflow do not allow fine-grained control of memory access.

We propose \sysname, a new attention algorithm that computes exact attention with far fewer memory accesses.
Our main goal is to avoid reading and writing the attention matrix to and from HBM.
This requires (i) computing the softmax reduction without access to the whole input (ii) not storing the large intermediate attention matrix for the backward pass.
We apply two well-established techniques to address these challenges.
(i) We restructure the attention computation to split the input into blocks and make several passes over input blocks, thus incrementally performing the softmax reduction (also known as \textbf{tiling}). (ii) We store the softmax normalization factor from the forward pass to quickly \textbf{recompute} attention on-chip in the backward pass, which is faster than the standard approach of reading the intermediate attention matrix from HBM.
We implement \sysname in CUDA to achieve fine-grained control over memory access and fuse all the attention operations into one GPU kernel.
Even with the increased FLOPs due to recomputation, our algorithm both \textbf{runs faster} (up to 7.6x on GPT-2~\citep{radford2019language}, Figure~\ref{fig:banner} right) and \textbf{uses less memory}---linear in sequence length---than standard attention, thanks to the massively reduced amount of HBM access.

We analyze the IO complexity~\citep{aggarwal1988input} of \sysname, proving that it requires $O(N^2 d^2 M^{-1})$ HBM accesses where $d$ is the head dimension and $M$ is the size of SRAM, as compared to $\Omega(Nd + N^2)$ of standard attention.
For typical values of $d$ and $M$, \sysname requires many times fewer HBM accesses compared to standard attention (up to 9$\times$ fewer, as shown in~\cref{fig:micros}).
Moreover, we provide a lower bound, showing that no exact attention algorithm can asymptotically improve on the number of HBM accesses over all SRAM sizes.

We also show that \sysname can serve as a 
useful primitive for realizing the potential of approximate attention algorithms by overcoming their issues with memory access overhead.
As a proof of concept, we implement block-sparse \sysname, a sparse attention algorithm that is 2-4$\times$ faster than even \sysname, scaling up to sequence length of 64k.
We prove that block-sparse \sysname has better IO complexity than \sysname by a factor proportional to the sparsity ratio.
We discuss further extensions to other operations (attention on multi-GPU, kernel regression, block-sparse matrix multiply) in~\cref{sec:discussion}.
We open-source \sysname to make it easier to build on this primitive.\footnote{\sysname code is available at \url{https://github.com/HazyResearch/flash-attention}}

We empirically validate that \sysname speeds up model training and improves model quality by modeling longer context. We also benchmark the runtime and memory footprint of \sysname and block-sparse \sysname compared to prior attention implementations.
\begin{itemize}[itemsep=0.1pt,topsep=0pt,leftmargin=*]
  \item \textbf{Faster Model Training.} \sysname trains Transformer models faster in wall-clock time. We train BERT-large (seq.\ length 512) 15\% faster than the training speed record in MLPerf 1.1~\citep{mattson2020mlperf}, GPT2 (seq.\ length 1K) 3$\times$ faster than baseline implementations from HuggingFace~\citep{wolf-etal-2020-transformers} and Megatron-LM~\citep{shoeybi2019megatron}, and long-range arena (seq.\ length 1K-4K) 2.4$\times$ faster than baselines.
  \item \textbf{Higher Quality Models.} \sysname scales Transformers to longer sequences, which improves their quality and enables new capabilities.
  We observe a 0.7 improvement in perplexity on GPT-2 and 6.4 points of lift from modeling longer sequences on long-document classification~\citep{dai2022revisiting}.
  \sysname enables the first Transformer that can achieve better-than-chance performance on the Path-X~\citep{tay2020long} challenge, solely from using a longer sequence length (16K).
  Block-sparse \sysname enables a Transformer to scale to even longer sequences (64K), resulting in the first model that can achieve better-than-chance performance on Path-256.
  \item \textbf{Benchmarking Attention.} \sysname is up to 3$\times$ faster than the standard attention implementation across common sequence lengths from 128 to 2K and scales up to 64K.
  Up to sequence length of 512, \sysname is both faster and more memory-efficient than any existing attention method, whereas for sequence length beyond 1K, some approximate attention methods (e.g., Linformer) start to become faster.
  On the other hand, block-sparse \sysname is faster than all existing approximate attention methods that we know of.
  
\end{itemize}


