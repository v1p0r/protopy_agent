\documentclass{article} %
\usepackage{iclr2026_conference,times}

\input{math_commands.tex}

\usepackage{graphicx}
\usepackage{wrapfig}
\usepackage[pagebackref=true,breaklinks=true,colorlinks=true,bookmarks=false]{hyperref}
\usepackage{url}
\definecolor{deepred}{HTML}{940000}
\hypersetup{linkcolor=deepred}
\hypersetup{urlcolor  = [rgb]{0.4,0.15,0.95}}
\hypersetup{citecolor=[rgb]{0.4,0.15,0.95}}



\usepackage{xspace}
\usepackage{slashed}
\usepackage{multirow}
\usepackage{makecell}
\usepackage{tabularx}
\newcolumntype{C}{>{\centering\arraybackslash}X}
\usepackage{algorithm,algpseudocode}
\usepackage{amsfonts}
\usepackage{booktabs}
\usepackage{siunitx}
\usepackage{amsthm}
\usepackage{wrapfig}
\usepackage{adjustbox}
\usepackage{scalerel}
\usepackage{enumitem}
\usepackage{siunitx}
\usepackage{listings}
\lstset{
  basicstyle=\ttfamily\small,
  breaklines=true,
  breakindent=0pt,   %
  xleftmargin=0pt,   %
  frame=single,
  columns=fullflexible,
  keepspaces=true,
}


\usepackage{epigraph}
\usepackage[align=center,shadow=true,shadowsize=4.5pt,nobreak=true,framemethod=tikz,skipabove=9.5pt,skipbelow=9pt,innertopmargin=5pt,innerbottommargin=5pt,innerleftmargin=5pt,innerrightmargin=5pt,leftmargin=1.5pt,rightmargin=1.5pt]{mdframed}
\usetikzlibrary{shadows}

\usepackage{placeins}


\usepackage{amssymb}
\usepackage{mathtools}
\usepackage{amsthm}
\usepackage{float}

\usepackage{color}
\usepackage{colortbl}
\definecolor{Gray}{gray}{0.94}

\usepackage{caption}

\usepackage{tocloft}
\usepackage[toc,page,header]{appendix}
\usepackage{minitoc}
\renewcommand{\ptifont}{\large \bf}
\renewcommand \thepart{}
\renewcommand \partname{}


\usepackage[T1]{fontenc}    %
\renewcommand{\captionlabelfont}{\footnotesize}





\title{\vspace{-4mm}
Agentic Design of Compositional Machines
}

\author{
\fontsize{9.5pt}{\baselineskip}\selectfont Wenqian Zhang\textsuperscript{1}~~~~~~Weiyang Liu\textsuperscript{2,*}~~~~~~Zhen Liu\textsuperscript{1,*,\textdagger}\\[.5mm]
\fontsize{9pt}{\baselineskip}\selectfont\textsuperscript{1}The Chinese University of Hong Kong (Shenzhen)~~~~\textsuperscript{2}The Chinese University of Hong Kong\\
\fontsize{9pt}{\baselineskip}\selectfont\textsuperscript{*}Equal Advising~~~~\textsuperscript{\textdagger}Corresponding Author~~~~~~{\tt\href{https://besiegefield.github.io/}{\textbf{besiegefield.github.io}}}
}


\newcommand{\fix}{\marginpar{FIX}}
\newcommand{\new}{\marginpar{NEW}}
\newcommand{\expt}{\mathop{\mathbb{E}}}
\newcommand{\Bignorm}[1]{\Bigl\lVert #1 \Bigr\rVert}
\newcommand{\norm}[1]{\left\lVert #1 \right\rVert}

\def\viz{\emph{viz}\onedot}
\usepackage{xspace}
\newcommand*{\eg}{{\it e.g.}\@\xspace}
\newcommand*{\ie}{{\it i.e.}\@\xspace}
\newcommand*{\cf}{{\it c.f.}\@\xspace}

\newcommand{\envname}{\textbf{BesiegeField}\xspace}

\newcommand{\zl}[1]{{\color{orange}{\bf\sf [Zhen: #1]}}}
\newcommand{\wy}[1]{\textcolor{cyan}{[Weiyang: #1]}}
\newcommand{\zwq}[1]{{\color{blue}{\bf\sf [Wenqian: #1]}}}

\newlength\savewidth\newcommand\shline{\noalign{\global\savewidth\arrayrulewidth
  \global\arrayrulewidth 1pt}\hline\noalign{\global\arrayrulewidth\savewidth}}

\theoremstyle{plain}
\newtheorem{theorem}{Theorem}%
\newtheorem{proposition}[theorem]{Proposition}
\newtheorem*{proposition*}{Proposition}
\newtheorem{lemma}[theorem]{Lemma}
\newtheorem{corollary}[theorem]{Corollary}

\theoremstyle{definition}
\newtheorem{definition}[theorem]{Definition}
\newtheorem{assumption}[theorem]{Assumption}

\newtheorem{remark}[theorem]{Remark}
\newtheorem*{remark*}{Remark}

\preprintcopy

\begin{document}


\doparttoc %
\faketableofcontents
\maketitle


\begin{figure}[h!]
  \vspace{-3.5mm}
  \centering
  \includegraphics[width=\linewidth]{figures/teaser_final_new.pdf}
  \vspace{-5mm}
  \caption{\footnotesize The task of compositional machine design is illustrated in our \envname environment. The figure shows a high-level sketch of the agentic workflow (w/ Gemini Pro 2.5), along with the resulting machines and their simulated performance. The design objective is to create a machine that throws boulders long distances.}
  \label{fig:teaser_candidate}
  \vspace{2.5mm}
\end{figure}

\begin{abstract}
\vspace{-1.5mm}
The design of complex machines stands as both a marker of human intelligence and a foundation of engineering practice. Given recent advances in large language models (LLMs), we ask whether they, too, can learn to create. We approach this question through the lens of compositional machine design: a task in which machines are assembled from standardized components to meet functional demands like locomotion or manipulation in a simulated physical environment. To support this investigation, we introduce \envname, a testbed built on the machine-building game Besiege, which enables part-based construction, physical simulation and reward-driven evaluation. Using \envname, we benchmark state-of-the-art LLMs with agentic workflows and identify key capabilities required for success, including spatial reasoning, strategic assembly, and instruction-following. As current open-source models fall short, we explore reinforcement learning (RL) as a path to improvement: we curate a cold-start dataset, conduct RL finetuning experiments, and highlight open challenges at the intersection of language, machine design, and physical reasoning.
\end{abstract}


\vspace{-4.5mm}

\setlength\epigraphwidth{5cm}
\epigraph{\emph{``Man is a tool-making animal.''}}{--- \textup{Benjamin Franklin}}
\vspace{-7mm}

\vspace{-1.5mm}
\section{Introduction}
\vspace{-1.5mm}

The history of human progress is, at its core, the history of machines, just as the ancient Greeks built the Antikythera mechanism to predict eclipses and Leonardo da Vinci envisioned machines to fly. Today, as large language models (LLMs) begin to approximate---and in some domains, surpass---human cognitive abilities, a natural question arises: 


\vspace{-1.5mm}
\begin{mdframed}[leftmargin=1em, rightmargin=1em]
\fontsize{9.3pt}{\baselineskip} 
\itshape Can computational models, like humans, conceive and create complex machines to achieve purposeful goals?
\vspace{-.35mm}
\end{mdframed}
\vspace{-3.75mm}

At the heart of this question lie two tightly coupled concepts: \textit{compositionality}, how parts are put together into assemblies, and \textit{functionality}, the tasks these assemblies perform as they interact with external forces or inputs. While foundation models are already capable of synthesizing 3D shapes and building mechanical parts with computer-aided design (CAD) models, it is the complex compositional structures, in which very different parts and components are orchestrated to smoothly move together, that realize a vast array of demands.
Just as a clock emerges from the composition of simple and standardized mechanical elements such as gears and flywheels, these same elements, when combined differently, can give rise to entirely different machines, such as a sewing machine. On the other hand, the same functionality may be realized by different part compositions, just as both cars and bicycles can transport a person from place to place. Put it concisely: \textit{composition is shaped by functionality, and functionality is realized through composition}. Since such compositional machines can be expressed programmatically, with types, placements and articulations of parts represented in structured code that LLMs can generate and manipulate, we formalize the above question as:

\vspace{-1.5mm}
\begin{mdframed}[leftmargin=1em, rightmargin=1em]
\fontsize{9.3pt}{\baselineskip} 
\itshape Can LLMs, given standardized mechanical parts and a reward function for the desired functionality, discover diverse spatial part compositions that maximize the reward and complete the task?
\vspace{-.35mm}
\end{mdframed}
\vspace{-3.75mm}

The question is not only about the pursuit of intelligence but also about the practice of engineering. Modern design pipelines are often long and costly, especially in large-scale projects where each iteration demands substantial resources. These projects accumulate vast collections of documents and blueprints, making it difficult to trace, retrieve, or reuse past design efforts. Much essential know-how is passed informally across teams and generations, and in many cases, never fully recorded and since forgotten. An automated machine design system could directly address these challenges.

Rather than merely mimicking patterns from historical designs, such a system should be agentic: capable of exploring the exponentially large design space, leveraging prior knowledge to create novel designs for new demands and constraints, and improving them through feedback. To investigate this concretely, we introduce \envname, an interactive environment built on the machine-design game of Besiege\footnote{\url{https://en.wikipedia.org/wiki/Besiege_(video_game)}}. The environment 
allows for construction of simple mechanical machines with standardized and semantic parts such as gears and wheels, and
supports customized physical scenarios in which LLM agents can test constructed machines and evaluate their dynamics and interactions.

Building on \envname, we benchmark state-of-the-art LLMs with different agent designs and strategies for selecting and placing basic mechanical elements to build machines for representative functional demands, a task we term \emph{compositional machine design}. Through these experiments, we empirically identify key capabilities required for this task: accurate spatial reasoning, high-level knowledge of design strategies, and instruction-following in spatial domains. Since only a few proprietary LLMs achieve satisfactory results, we further investigate how reinforcement learning (RL) can improve the performance of open-source LLMs. To this end, we curate a small machine design dataset to cold-start RL finetuning, perform exploratory RL experiments, and highlight key challenges that chart directions for future research. In summary, our contributions are listed below:

\vspace{-1mm}
\begin{itemize}[leftmargin=*,nosep]
\setlength\itemsep{0.35em}
    \item We introduce and formalize the task of \emph{compositional machine design}, where machines are assembled from standardized parts to achieve functional goals.
    \item We present \envname, an interactive environment 
    that enables LLM agents to construct, simulate, and evaluate compositional machines in customized physical scenarios.
    \item We systematically benchmark state-of-the-art LLMs and different agentic workflow designs on representative machine-design tasks.
    \item We explore RL finetuning of LLMs on this task, for which we curate a cold-start dataset, conduct experiments, and highlight the key challenges.
\end{itemize}

\vspace{-2mm}
\section{Compositional Machine Design}
\vspace{-2mm}




Full machine design involves many coupled elements: geometry, statics and dynamics, demand analysis, failure modes, safety, and even legal constraints~\citep{beitz1996engineering,wong2025llm}. To isolate a tractable subproblem, we focus on the structural composition of machines: how standardized parts are spatially arranged and mechanically linked to produce functional behavior. We refer to this task, introduced in the previous section, as \emph{compositional machine design}. It captures two essential components: (i) the static geometry of a machine as a part-based assembly, and (ii) its compatibility with functional demands, typically assessed through physical simulation. This abstraction omits considerations such as manufacturing constraints, material properties, or domain-specific regulations, but retains the core spatial and behavioral reasoning challenges relevant to design.

This special task of compositional machine design mirrors challenges found in other exploration domains. For example, automatic theorem proving involves a compositional and exponentially large action space, while electronic design automation (EDA) for chip layouts requires spatial reasoning to place components of varying shapes under spatial constraints (albeit in a more regular and grid-constrained fashion than mechanical parts in machines). A unique challenge in machine design, however, is its dependence on diverse long-horizon behaviors, both autonomous and non-autonomous, within an environment. Specifically, a machine may behave differently when operated in different ways (\eg, a bicycle when pedaled versus when braking) or under different external conditions (\eg, driving a car in sunny versus rainy weather). Similarly, many sophisticated machines cannot function without appropriate control policies, as exemplified by aircraft that rely on fly-by-wire systems to stabilize their inherently unstable aerodynamic configurations (which would otherwise be unflyable by a human pilot alone). A key open problem is therefore how to account for the interplay among physics, control policy, and compositional structure in machine design.


It is worth noting that, unlike in math theorem proving where one valid proof often suffices (even though multiple proofs may still be valued), design domains typically require generating a diverse set of candidate solutions. This diversity is essential to (i) differentiate products, (ii) adapt to unpredictable market demands, and (iii) account for uncertainty in real-world testing and deployment. Consequently, the task places greater emphasis on diversity, and a model for compositional machine design should function more like a generative model than a simple reward maximizer.

\vspace{-2mm}
\section{\envname: Playground for Compositional Machine Design}
\vspace{-2mm}


Studying the full problem of compositional machine design is challenging, as it involves the coupling of many interacting factors.
We therefore focus on a minimalist, component-level setting in which machines are constructed primarily from cuboid primitives with clear functional semantics, together with a small set of specialized exceptions, and operate under a shared control policy in an environment governed by rigid-body and elastic mechanics. This abstraction allows us to properly benchmark the capabilities of existing LLMs and to assess the upper bounds, potential, and challenges of agentic systems and RL algorithms.

To this end, we create \envname, an interactive environment adapted from the machine-building game Besiege, in which players design medieval machines to complete tasks such as destroying castles. 
Powered by the built-in physics engine, \envname supports physical simulation of mechanical systems such as vehicles and catapults in user-customized environments with terrains, obstacles, external forces (\eg, wind and gravity), and co-existing agents. The environment provides nearly 80 types of building blocks (examples illustrated in Fig.~\ref{fig:block_intro}), including passive ones like drills and logs, and powered ones like powered cogs and wheels.
Machines are constructed by sequentially attaching new parts to vacant and attachable faces of existing blocks, starting from a root block and thus forming a ``construction tree'' (indeed a directly acyclic graph (DAG), in the sense of operation orders; one block can has two parents in the DAG; the actual structures may contain loops).
Powered blocks can receive control commands, allowing machines to be operated precisely. During simulation, complete state information (e.g., the position and velocity of each block in the constructed machine) can be recorded for model feedback. Finally, the environment supports custom modifications and can be extended with additional block types and richer physics (\eg, simple fluid simulation). Further details are explained in Appendix \ref{appendix:env}.




\envname is unique in balancing real-world geometry and physics, part-level semantics, and simple compositional rules. Block-stacking environments like LEGO~\citep{fan2022minedojo} and Minecraft~\citep{fan2022minedojo,pun2025generating} allow intuitive combinatorial assembly but do not natively provide realistic physical simulation and rely on generic blocks with limited semantic meaning. CAD modeling~\citep{li2025cad} captures fine-grained geometry and interactions, but its complexity makes rules cumbersome and sequences prohibitively long. By contrast, \envname uses semantically meaningful parts with cuboid-like construction rules—supporting realistic physics while remaining abstract enough for tractable composition. This calibrated balance enables the study of compositional creativity and geometric reasoning at a level of difficulty that both differentiates algorithms and permits rapid experimentation. Moreover, unlike prior environments, \envname supports machine destruction, adding durability and failure analysis to the design space.




\vspace{-2mm}
\section{Benchmarking LLMs for Compositional Machine Design} \label{sec:agentic}
\vspace{-1.5mm}


\subsection{Benchmark Settings}
\vspace{-1.5mm}

\textbf{Representative target machines and tasks.} To benchmark and characterize the performance of different LLMs for agentic compositional machine design, we consider two conceptually simple yet representative target machines to build: \textit{car} and \textit{catapult} as shown in Fig.~\ref{fig:task_demo}. While success in both requires understanding part semantics and structural syntax, \textit{car} building primarily tests static relational reasoning, such as enforcing correct part orientations, symmetry, and stability; in contrast, \textit{catapult} building challenges models with dynamic relational reasoning, where parts must coordinate over time to produce causal mechanical effects. Moreover, the two tasks are simple enough to be constructed with only a few blocks so that they fit within the LLM’s context window, yet complex enough to require explicit reasoning about construction strategies and causal dependencies. We evaluate the performance of \textit{cars} and \textit{catapults} by their moving distance and their throwing distance (\ie, the moving distance of the stone), respectively, towards a fixed and given direction. During each simulation, the generated machine will be placed at a designated position, and the active parts will be powered after a few seconds. As there can be reward hacking issues, for \textit{catapults} experiments we surround the designated machine placement position with moderate-height walls. More details about the target machines, rewards, and environments can be found in Appendix~\ref{appendix:env}. 


\begin{figure}[t!]
  \centering
  \vspace{-3mm}
  \includegraphics[width=\linewidth]{figures/environment_intro_cropped.pdf}
  \captionsetup{font=footnotesize} %
  \vspace{-6.5mm}
  \caption{\footnotesize Demonstration of the machine design tasks in our experiments. (Left: \textit{car}; Right: \textit{catapult}).}
  \label{fig:task_demo}
  \vspace{-1.5mm}
\end{figure}

\vspace{-.5mm}

\textbf{Machine representations.} In \envname, the default XML representation records all blocks with global 3D positions and uses a built-in algorithm to recover connections. Such a representation, however, does not well encode machine structures. Instead, we propose a parsimonious representation aligned with the game’s building logic, based on pairwise relative attachment relationships (i.e., how one part is rotated and attached to another). Details are explained in Appendix~\ref{sec:tree_rep}.
\vspace{-.5mm}

\begin{figure}[t!]
  \centering
  \includegraphics[width=0.98\linewidth]{figures/xml_intro_v2_cropped.pdf}
  \vspace{-2.5mm}
  \captionsetup{font=footnotesize} %
  \caption{\footnotesize Demonstration of the default XML representation and our construction tree representation. Parent block info is in blue and child info is in red.}
  \label{fig:XML-intro}
  \vspace{-4.5mm}
\end{figure}

\textbf{Performance metrics.} We evaluate our agentic systems using the following quantitative metrics: 1) \textit{file validity rate}, the proportion of generated JSON files that can be successfully parsed into machine construction trees; 2) \textit{spatial validity rate}, the proportion of generated machines that are free from self-collisions; 3) \textit{machine validity rate}, the proportion of machines that satisfy both file and spatial validity; 4) \textit{mean and maximum simulation scores}, the average and highest rewards achieved by generated machines in the environment.

\textbf{Environment feedback.} For the simple target machines \textit{car} and \textit{catapult}, we consider environment feedback within a time window of 5 seconds that is long enough to characterize their designated functionalities. Specifically, for \textit{car} we consider maximum speed and driving distance; for \textit{catapult}, we consider boulder throwing distance and maximum height. We also record the machines' global orientation and broken parts information (if any). Details are elaborated in Appendix~\ref{sec:env_feedback}.



\begin{figure}[t!]
  \centering
  \vspace{-2mm}
  \includegraphics[width=0.98\linewidth]{figures/gemini_cot_example_v3.pdf}
  \vspace{-2mm}
  \captionsetup{font=footnotesize} %
  \caption{\footnotesize Example CoT of inspector agents (w/ Gemini 2.5 Pro). Blue text highlights the moderate capability of LLMs in spatial reasoning and imagined physical simulation.}
  \label{fig:gemini-cot-example}
  \vspace{-1mm}
\end{figure}

\begin{figure}[t!]
  \centering
  \includegraphics[width=0.98\linewidth]{figures/o3_cot_example_v4_green_cropped.pdf}
  \vspace{-2mm}
  \captionsetup{font=footnotesize} %
  \caption{\footnotesize Example CoT of inspector agents (w/ OpenAI o3). Red text highlights reasoning errors.}
  \label{fig:o3-cot-example}
  \vspace{-4.5mm}
\end{figure}






\vspace{-1.25mm}
\subsection{Agentic Workflow Design}
\vspace{-1.25mm}


\textbf{Single-agent setting.} We first benchmark if a single LLM agent alone is capable of completing the task. Specifically, one LLM agent is provided with the environment description, the available machine components, the assembly syntax, and the functional requirements (\eg, moving an object forward). The agent generates a chain-of-thought (CoT; \citet{wei2022chain}) to reason about what is needed and why, and then derives an abstract plan (e.g., connecting a lever to a container with a boulder). This plan is later translated into the construction tree representation.

\textbf{Iterative editing.} Because compositional machine design requires both low-level spatial reasoning and high-level ideation, a single agent rarely produces satisfactory machines. We therefore also design an iterative editing workflow that involves three major agents: 1) \emph{designer}, which produces an initial plan from the environment description, the available machine components, the assembly syntax, and the functional requirements; 2) \emph{refiner}, a self-critic agent that which evaluates a draft against requirements and constraints and proposes multiple candidate revisions at each step; 3) \emph{environment querier}, an agent that runs machine simulation and summarizes the environment feedback, in the way that it always provides global information such as machine orientation throughout the trajectory but selectively reports the feedback on specific blocks (\eg, position and speed) for further machine refinement. The workflow begins with a draft from the designer that is later critiqued by an \textit{inspector}, which assess the designed machine in an abstract fashion, then polished once by a refiner. The design then undergoes a fixed number of iterations, each consisting of one querier and one refiner step. At refiner stages, multiple candidates are generated for running Monte Carlo tree search (MCTS; \citet{coulom2006efficient}). The best design found in this search process is selected as output.



\begin{wrapfigure}{r}{0.55\linewidth}
  \vspace{-5mm}
  \centering
  \includegraphics[width=\linewidth]{figures/besiegefield_pipeline_compressed.pdf}
  \vspace{-6mm}
  \caption{\footnotesize Our agentic machine design workflow.}
  \label{fig:multi-agent-detail}
  \vspace{-3mm}
\end{wrapfigure}

\textbf{Hierarchical construction.} Inspired by typical human design processes as well as recent designs of agentic systems \citep{xiao2025verbalized,teng2025atom,zhang2024aflow}, we introduce a meta-designer agent that first analyzes the requirements and constraints, and then constructs a high-level blueprint of the major functional blocks (\eg, the suspension system) and their interconnections. With this blueprint in place, we adopt an autoregressive strategy to build the machine block by block: 1) we begin with the first functional block and dispatch the job to eight parallel builder agents; 2) the valid designs from this stage are evenly distributed to another eight builder agents to construct the second block; and 3) the process iterates in this manner until the entire machine is assembled. Empirically, we find that the meta-designer typically decomposes a machine into three to four functional blocks.


\begin{figure}[t!]
  \centering
  \vspace{-2mm}
  \includegraphics[width=\linewidth, clip, trim={0 0.7cm 0 0}]{figures/LLM_Comp_v4_cropped.pdf}
  \captionsetup{font=footnotesize} %
  \vspace{-6mm}
  \caption{\footnotesize Machines produced by agentic systems with different LLMs (Top: \textit{car}; Bottom: \textit{catapult}).}
  \label{fig:test-time-scaling-comp}
  \vspace{-1.5mm}
\end{figure}



\begin{table}[t!]
  \scriptsize
  \centering
  \setlength{\abovecaptionskip}{5pt}
\setlength{\belowcaptionskip}{-1pt}
  \setlength{\tabcolsep}{3pt}
  \renewcommand{\arraystretch}{1.1}
  \newcommand{\cgr}[1]{\textcolor[rgb]{.329,.51,.208}{\textbf{#1}}}
  \newcommand{\cre}[1]{\textcolor[rgb]{1, 0, 0}{\textbf{#1}}}
  \renewcommand{\pm}{\mathbin{\text{±}}}
  \begin{tabularx}{\textwidth}{l*{9}{>{\centering\arraybackslash}X}}
    \multirow{2.4}{*}{Models}
    & \multicolumn{3}{c}{Single-agent} & \multicolumn{3}{c}{Iterative Editing}& \multicolumn{3}{c}{Hierarchical Design} \\
    & Mean & Max&Std & Mean & Max&Std& Mean & Max&Std \\
    \shline
    \multicolumn{6}{l}{~~~~\textbf{\textit{``Catapult'' Task}}} \\
    Gemini 2.5 Pro
       &2.30&9.0 &3.86 
       &4.67&\bf 21.95&8.68
       &\bf 9.83&\bf 18.19& 8.35
       \\
    OpenAI o3
      &2.87&5.22 &1.96
      &\bf 9.14&14.01&3.71
      &2.00&11.11&3.98
       \\

    Qwen3-Coder-480B-A35B
       &1.75&9.24&3.17 
       &5.10&12.02&5.54
       &3.90&6.52&2.54
       \\

    Doubao Seed 1.6
       &3.18&8.2&2.99
       &4.82&9.10&3.41
       &1.73&4.76&2.39
       \\

    Claude Opus 4
       &1.19&4.82&2.21
       &1.18&4.91&2.18
       &2.27&9.32&4.22
       \\

    DeepSeek-V3
       &\bf 3.50&4.86&2.17
       &3.07&5.24&2.55
       &2.41&4.93&2.58
       \\

    Kimi K2
       &2.57&\bf 9.05&3.72
       &2.82&11.39&5.23
       &5.39&12.02&5.16
       \\

    Llama 4 Scout 17B 16E
       &3.18&5.64&1.95
       &1.28&5.94&2.41
       &3.59&11.83&4.15\\
    \hline
    \multicolumn{6}{l}{~~~~\textbf{\textit{``Car'' Task}}} \\
    Gemini 2.5 Pro
       &\bf 33.96&\bf 40.85&6.73
       &\bf 34.34&\bf 41.66&13.96
       &\bf 29.96&\bf 41.52&7.78\\
    OpenAI o3
      &15.28&32.08&8.97
      &14.34&35.08&11.79
      &28.39&36.18&11.01\\
    Qwen3-Coder-480B-A35B
       &8.87&11.50&4.46
       &15.24&28.95&13.12
       &12.59&34.05&10.78\\
    Doubao Seed 1.6
       &3.51&9.40&4.85
       &8.11&10.04&3.58
       &18.75&26.02&4.38\\
    Claude Opus 4
       &9.83&12.98&1.28
       &8.07&28.04&12.48
       &14.56&38.67&20.69\\
    DeepSeek-V3
       &9.06&10.53&3.68
       &8.23&18.84&7.12
       &17.92&31.94&12.85\\
    Kimi K2
       &1.75&8.09&2.80
       &14.36&28.34&9.47
       &1.94&14.99&5.48
       \\
    Llama 4 Scout 17B 16E
       &0.02&0.03&0.01
       &3.04&12.76&5.23
       &1.55&2.00&0.32
       \\
  \end{tabularx}
  \caption{\footnotesize Quantitative results of agentic systems with different LLMs.}
  \label{tab:agentic}
  \vspace{-5mm}
\end{table}

\vspace{-2mm}
\subsection{Key Empirical Observations}\label{sec:agent_results}
\vspace{-1mm}

\textbf{General observations.} 
We find compositional machine design to be a challenging task for LLMs (Fig.~\ref{fig:test-time-scaling-comp} and Table~\ref{tab:agentic}), though not intractable: Gemini 2.5 Pro can consistently construct visually sensible machines with non-trivial performance. We find no evidence that reasoning models outperform non-reasoning ones, suggesting the main bottleneck lies in LLMs’ limited 3D understanding and/or in-context learning. We also find that LLMs, especially reasoning models, still exhibit some spatial and physical reasoning as exemplified by the CoT from Gemini Pro 2.5 (Fig.~\ref{fig:gemini-cot-example}), much like a world model in text space.

\textbf{Failure patterns.} We identified common failure patterns in LLM-generated machines (Fig.~\ref{fig:failure-mode}): 1) \textit{incorrect part orientations}; 2) \textit{incorrect part placements}, where parts attach to wrong parents; 3) \textit{instruction-following failures}, where elements of the high-level blueprint are not strictly observed; 4) \textit{flawed high-level reasoning}, where LLMs fail to recognize correct physics or essential components.



\textbf{Effect of environment feedback.} It is unsurprising that with the more environment feedback the agents receive, the better performance of generated machines improve in general (Table~\ref{tab:rl_env_feedback}).


\textbf{Effect of edit history.} We find that edit histories are generally helpful in decreasing the number of failure attempts in creating valid machines (Table ~\ref{tab:edit_history}), which underscores the importance of longer context window of base models for efficient exploration.


\textbf{Hierarchical design.} We observe the mean performance improves with hierarchical design only when the abstract-level reasoning on blueprints is reliable, as shown by the performance of Gemini 2.5 Pro. In the meantime, consistent with the intuition that hierarchical design is more structured and principled, it generally yields lower variance in obtained scores.


\textbf{Effect of CoT reasoning.} As shown in Fig.~\ref{fig:failure-mode}, LLMs often fail to faithfully translate high-level machine design plans in their CoT into semantically and geometrically consistent machine construction trees. To better assess the impact of CoT reasoning on high-level design, we feed the CoT generated by Gemini 2.5 Pro (the best-performing model) to other LLMs, prompting them to directly output construction trees. The resulting machines generally show improved performance (Fig.~\ref{fig:LLM-feed-gemini-cot}) and highlight the critical role of high-level semantic reasoning in machine design. 

\textbf{CoT-machine correspondence.} Though the CoT often provides a reasonably high-level blueprint, agents may still generate machines that deviate from the intended structure (Fig.~\ref{fig:failure-mode}). We hypothesize that this misalignment is a key reason many LLMs struggle to build better machines.




\textbf{Machine representation.} We experiment with a coordinate-only representation derived from the default XML (Appendix~\ref{sec:machine_rep}) and our construction tree representation. Results show that the coordinate-only representation performs significantly worse (Table~\ref{tab:rep_comparison}), implying that explicit structural information is necessary for LLM understanding.

\textbf{3D information.} 
We observe that (Table~\ref{tab:abl:machine-3d-info}) the performance generally improves when we also feed parsed 3D information into the context of LLMs, which implies that LLMs are less capable of understanding relative spatial relationship (\eg, construction trees).


\vspace{-2mm}
\section{Towards Machine Design through Reinforcement Learning}
\vspace{-2mm}

Although agentic systems show promise in compositional machine design, simply scaling system size is unlikely to be economical, as errors compound rapidly. Like humans who internalize experience, LLM agents should consolidate new knowledge into weights. We thus explore reinforcement learning with verifiable rewards (RLVR) in \envname to develop machine-design capabilities.

\vspace{-2mm}
\subsection{Experimental Settings}\label{sec: setting}
\vspace{-1.5mm}


\textbf{Cold-start finetuning and dataset curation.} Following recent RLVR practices~\citep{lambert2024tulu,yue2025does,zhu2025surprising}, we curated a small dataset to cold-start LLMs by aligning their reasoning process with expert CoT. Specifically, we collected textual descriptions of machine functionalities from Besiege player communities and prompted Gemini 2.5 Pro to generate corresponding machines. After filtering out invalid generations, we obtained 9,984 valid machine–CoT pairs. We then used this dataset to perform supervised finetuning on Qwen-2.5-14B-Instruct for 12 epochs. Additional training details are provided in Appendix \ref{sec:cold_start_details}.


\textbf{Reward design.} 
We use the reward $R = \texttt{is\_valid} \times \texttt{performance}$ where $\texttt{is\_valid}$ indicates whether constraints are satisfied (Appendix~\ref{sec:reward setting}). For \textit{car}, $\texttt{performance}$ is the maximum travel distance; for \textit{catapult}, it is the product of the boulder’s maximum height and distance, penalizing solutions that are extreme in only one dimension.

\textbf{RL finetuning settings.} 
We finetune agents specialized in building a single type of machine (either \textit{car} or \textit{catapult}), making our setup closely aligned with one-shot RLVR~\citep{wang2025reinforcement} where a single prompt is used throughout the RL process. We adopt group relative policy optimization (GRPO; \citet{shao2024deepseekmath}) with LoRA parametrization~\citep{hu2022lora}  (rank 64) and mixed-precision training to finetune the cold-started model. We evaluate both the standard GRPO advantage estimator and the pass@k variant~\citep{tang2025optimizing}. In the latter case, due to the implementation of the RLVR framework verl~\citep{sheng2025hybridflow}, the number of rollouts is set equal to $k$. Each experiment is run for 400 iterations on 8 A100 GPUs with per-GPU batch size of 1 and gradient accumulation of 8. We apply KL regularization with strength 0.001 to encourage the model to remain close to its initialization.


\begin{table}[t!]
  \scriptsize
  \centering
  \setlength{\tabcolsep}{2.9pt}       %
  \renewcommand{\arraystretch}{1.25} %
  \newcommand{\cgr}[1]{\textcolor[rgb]{.329, .51, .208}{\textbf{#1}}} %
  \newcommand{\cre}[1]{\textcolor[rgb]{1, 0, 0}{\textbf{#1}}}          %
  \renewcommand{\pm}{\mathbin{\text{±}}} %

\begin{tabularx}{\textwidth}{
  l|
  *{3}{>{\centering\arraybackslash}X}| 
  *{3}{>{\centering\arraybackslash}X}
}
\multirow{2}{*}{Models} 
 & \multicolumn{3}{c|}{\textit{Catapult}}
 & \multicolumn{3}{c}{\textit{Car}} \\
 & Validity Ratio & Mean Score & Max Score
 & Validity Ratio & Mean Score & Max Score \\
\shline
Qwen2.5-14B-Instruct           & 11/50 & 0.06 &  2.41 & 46/50 & 4.97 & 19.10 \\
Qwen2.5-14B-Instruct + Cold-Start     &  9/50 & 0.11 &  5.54 & 40/50 & 4.67 & 20.23 \\
Qwen2.5-14B-Instruct + RL      & \textbf{12}/50 & 0.13 &  5.92 &   41/50 &  3.72  &   24.08  \\
Qwen2.5-14B-Instruct + Cold-Start + RL& 11/50 & \textbf{0.14} &  \textbf{7.14} &   \textbf{42}/50  &  \textbf{5.05}  &   \textbf{45.72}  \\
\end{tabularx}
\vspace{-2.5mm}
\caption{\footnotesize Results of RLVR post-training in \envname. We use Qwen2.5-14B as the backbond LLM.}
\label{tab:rl_general}
\vspace{-3mm}
\end{table}



\vspace{-1.5mm}
\subsection{Main Results and Observations}
\vspace{-1mm}

\textbf{General results.} As shown in Fig.~\ref{fig:RL-catapult-valid-rate}, RL finetuning can generally improve the mean performance, mostly by increasing the percentage that machines are valid (including file validity, machine validity and satisfaction of minimum performance threshold). In the meantime, we also find that the maximum reward increases in our best setting. Similar to observations in many other RLVR settings, the entropy of the output distribution quickly drops even with regularization.

\textbf{Pass@k advantage vs. Pass@1 advantage.}
Since we eventually care about the best performing designs, especially given the low validity rate, our default setting adopts Pass@k advantage estimator. Indeed, Pass@k finetuning is more likely to discovery promising machine designs (Fig.~\ref{fig:RL-catapult-max-score}).


\begin{wrapfigure}{r}{0.4\linewidth} %
  \centering
  \vspace{-4mm} %
  \includegraphics[width=\linewidth]{figures/14b_model_sample_v4_cropped.pdf}
  \captionsetup{font=footnotesize}
  \vspace{-5.5mm} %
  \caption{\footnotesize Designs at RL finetuning stages.}
  \vspace{-3mm} %
  \label{fig:rl_evolution}
\end{wrapfigure}

\textbf{Evolution of generated machines during finetuning.} In Fig.~\ref{fig:rl_evolution}, we qualitatively examine how models refine their designs over the course of finetuning. We observe that models typically make detail-level adjustments, such as shifting part positions, while keeping the same high-level design strategy rather than exploring alternative strategies. Although these strategies are often reasonable, the models struggle to find precise configurations that enable smooth coordination among parts. This precision is especially critical for sophisticated mechanisms like catapults to function properly.

\textbf{Cold-start.}
Not surprisingly, we find that cold-start alone does not enable models to produce satisfactory designs, and that finetuning on the cold-start model is better than on the base model (Table~\ref{tab:rl_general}).



\vspace{-1.5mm}
\section{Discussions and Intriguing Insights}
\vspace{-1.5mm}

\textbf{Capabilities for compositional machine design.} Although tasks such as visual understanding and generation also depend on spatial, physical, and semantic reasoning, compositional machine design introduces unique requirements for LLM capabilities. Without precise spatial placement of machine parts, a design may fail to function correctly; a gear train, for example, will not transmit rotation if the gears are misaligned. Since the design process is typically hierarchical, successful LLMs must be able to accurately translate high-level blueprints into detailed geometric designs. In addition, machine design spans both concept-level reasoning and detailed specification. This dual demand often leads to large design documents and calls for a form of “visual reasoning” expressed through text, similar to what has been studied in LLMs applied to scalable vector graphics (SVG) and CAD models~\citep{qiu2025sgpbench,alrashedy2025generating}. Multimodal reasoning is also important because effective machine design typically relies on integrating textual descriptions with visual or schematic representations. In this work, however, we focus only on pure LLM-based reasoning to isolate and analyze its capabilities for compositional machine design.




\textbf{Challenges in agentic machine design systems.} The task of machine design faces similar challenges found in agentic systems in domains such as legal services and other knowledge-intensive fields. A key difficulty is the highly varied requirements and domain knowledge of different customers. To address this, LLMs need to acquire task-specific knowledge through in-context learning or finetuning. In addition, the complexity of design tasks often requires multiple agents to coordinate, and such pipelines can suffer error accumulation when the base LLM lacks sufficient capability.

\textbf{Exploration in machine design space.} Different from tasks such as theorem proving, the goal of compositional machine design is to discover structures that more effectively achieve desired functionalities. Rather than reusing existing solutions, a practical design agent should be able to propose novel strategies, structural layouts, and part specifications as machine complexity increases. Meeting this requirement calls for RL finetuning methods that prevent models from collapsing into a narrow set of strategies and structures, which recent methods aim to alleviate~\citep{zhu2025flowrl,chen2025pass, cui2025entropy,cheng2025reasoning,liu2025nablagfn}. This demand is closely related to continual RL~\citep{schwarz2018progress}, since finetuned LLMs must avoid catastrophic forgetting, maintain its reasoning ability, and consolidate learned strategies, which is particularly important because large-scale machine design datasets are rare and commercially infeasible to collect.


\vspace{-1mm}
\section{Related Work and Concluding Remarks} 
\vspace{-1mm}

\textbf{3D graphics codes for generative modeling.} There is a long history in 3D asset generation and engineering design of representing the construction of a target instance as a program or sequence of operations in a domain-specific language~\citep{ritchie2023neurosymbolic,sun20253d,deng2022unsupervised}, which we refer to here as 3D graphics codes~\citep{qiu2025sgpbench, chen2025symbolic}. Unlike geometric representations such as point clouds or meshes, these codes describe objects at a higher semantic level, capturing part composition, design constraints, and user operations in modeling software. 
Similar to programming languages, 3D graphics codes are inherently discrete and are typically generated with autoregressive models trained from scratch~\citep{yuan2024cadtalk} or with LLMs finetuned on curated datasets~\citep{kulits2025re,chen2025sar3d}. Much of the existing work centers on CAD scripts for individual parts~\citep{wu2023cad, alrashedy2025generating,li2025cad} or Blender macros for single assets~\citep{huang2024blenderalchemy}. Whereas recent studies on LEGO assemblies~\citep{pun2025generating}, Minecraft structures~\citep{fan2022minedojo,liu2024odyssey}, and procedural scene generation~\citep{sun20253d,chen2025symbolic,jones2025shapelib,yuan2024cadtalk} introduce richer compositionality, they still fall short of the task of compositional machine design, which requires assemblies that both function under physical laws and exhibit the precise geometry of real objects.


\textbf{LLM agents.} LLM agents are language models organized to operate in iterative loops of perception and action~\citep{yao2023react,minaee2024large,hu2024scenecraft}. They interact with external tools~\citep{schick2023toolformer,liu2024toolnet,kim2024leveraging,qin2024toolllm}, respond to signals from simulated or real environments~\citep{savva2019habitat,shridhar2020alfworld}, incorporate self-reflection to refine their outputs~\citep{hu20243d,alrashedy2025generating,shinn2023reflexion,yu2025generating}, and are commonly organized into multi-agent systems that coordinate roles and exchange information
~\citep{li2023camel,chen2024agentverse,zhang2024aflow}
. These designs move beyond one-shot text generation and establish LLMs as adaptive decision makers capable of long-horizon reasoning. Approaches that introduce search over possible solutions~\citep{yao2023tree,putta2024agent,koh2024tree} or reflection on prior attempts~\citep{besta2024graph,deng2024multi,renze2024self,xiao2025verbalized,yu2025generating} have enabled progress on increasingly complex tasks. 
LLM agents have already been used in design tasks such as code synthesis~\citep{gao2023pal,novikov2025alphaevolve,madaan2023self}, CAD design~\citep{alrashedy2025generating} and game environments~\citep{wang2023voyager,fan2022minedojo}. 
Partially inspired by these developments, \citet{makatura2023can} proposed a prototypical agent-based design framework that generates mechanical structures from text prompts. Their system treats structure generation as a one-shot process and delegates the search for optimal geometric and physical parameters to external optimization tools. In contrast, our work with \envname explores how LLM agents can directly and iteratively bridge compositional structures to functional goals, framing design as a process of reasoning and adaptation with both accurate simulation and intuitive physics.


\textbf{Reinforcement learning with verifiable rewards (RLVR).} Recent studies indicate that, by running RL finetuning with verifiable rewards from simulators or verifiers, reasoning abilities emerge~\citep{shao2024deepseekmath,guo2025deepseek,bai2022constitutional}, even when single prompt is used during finetuning~\citep{wang2025reinforcement}.
Yet, many methods exhibit loss of diversity as output entropy collapses during reinforcement learning and thus do not fully enable LLMs to explore novel solutions. Examples of mitigation methods include explicit entropy or KL regularization~\citep{cui2025entropy,ouyang2022training}, Pass@k training~\citep{tang2025optimizing,chen2025pass}, and distribution-matching objectives like generative flow networks~\citep{zhu2025flowrl,hu2023amortizing}. 
\envname provides verifiable rewards and thus enables direct application of RLVR to compositional machine design.



\textbf{Concluding remarks}. We introduced \textit{compositional machine design}, a simplified yet challenging task that reflects core aspects of real-world machine design. To evaluate LLM performance on this task, we developed \envname, an interactive environment based on the game Besiege. Our results with agentic systems and reinforcement learning demonstrate that LLMs hold promise for solving this problem. While we did not exhaustively explore all designs or integrate multi-modal information, our findings underscore the need to advance fundamental LLM algorithms and capabilities, and point toward exciting future directions in machine design.





    
    
    
    


\section*{Acknowledgment}

We sincerely thank the developers of Besiege for creating the game and fostering an open and vibrant community, without which our exploration of this exciting idea would not have been possible.

\bibliography{iclr2026_conference}
\bibliographystyle{iclr2026_conference}

\section{Hyperparameter and Training Details} \label{appendix:hyper-param}
Our hyper-parameter settings are shown in \cref{tab:appendix-hyperparam}. For GLUE task fine-tuning, we follow the hyper-parameter setting of CoFi~\citep{xia_structured_2022}, separating the tasks into big (MNLI, SST2, QNLI, QQP) and small (MRPC, CoLA, RTE, STSB) based on the dataset size. For instruction tuning on the Alpaca dataset, we train the pruned model for 15 epochs after the pre-tuning pruning process to make sure they converge. However, in practice, such training epochs can be reduced. To adaptively increase the tuning parameters in the {\lmabbr}, at the start of fine-tuning, we initialize adapter ranks to 8, with salient layers' ranks linearly increased. The scaling factors are set as 2 statically. Since evaluating billion-level LLaMA models during instruction tuning with all evaluation tasks would be time-consuming, we did not do the TTA evaluation as small models. We do not conduct any hyper-parameters search for any training for fair comparison.

% Please add the following required packages to your document preamble:
% \usepackage{booktabs}
\begin{table}[htbp]
\centering
\begin{tabular}{@{}llllll@{}}
\toprule
Hypeparameter  & GLUE-small & GLUE-big & SQuAD & CNN/DM & Alpaca \\ \midrule
Learning rate  & 2e-4       & 2e-4     & 2e-4  & 1e-4   & 1e-4   \\
Batch size     & 32         & 32       & 32    & 16     & 32     \\
Epochs         & 40         & 40       & 40    & 16     & 15     \\
Distill epochs & 20         & 20       & 20    & 6      & -      \\ \bottomrule
\end{tabular}
\caption{Hyperparameters used in {\ourmethod} experiments}
\label{tab:appendix-hyperparam}
\end{table}

% \subsection{Training Details}
% Unlike the existing fine-pruning process that keeps the model size fixed with masks to control the model's sparsity throughout training, we gradually prune out structures and shrink the model size during training for training acceleration and memory savings.
% \todo{define sparsity again -> pruning parameters / total parameters; range of tuning parameter ratio (peak and average)}
% \qq{put these into implementation details and add to appendix}
% For task performance evaluation, we report the accuracy of SST2 and MNLI on their dev sets, F1 score on SQUAD V2 dev set, ROUGE 1/2/L scores on the CNN/DM test set, and BLEU score on WMT en-ro test set. As for the instruction-tuning setup, we report the zero-shot multiple choice accuracy of the model on MMLU, while the GPT-4 annotated win rate of the model's generation on AlpacaEval. 
% \hanna{A section or subsection on training details is missing. there are a lot of hyperparameters that you need to list here or briefly mention and then refer to appendix. Also, there are some details that you've mentioned in the method and I proposed to move here.  }
When pruning {\lmabbr}s with {\ourmethod}, following \citep{xia_structured_2022}, we first prune and train the {\lmabbr} with the self-distillation objective, and then fine-tune the pruned {\lmabbr} to recover its end-task performance. Given $T$ pruning training steps in total, we set a pre-determined target sparsity $\gamma_T$ (defined as the ratio of pruned parameter size to the total parameter size) and use cubic scheduling to control the {\lmabbr} parameter size, where $\gamma_t = \gamma_T + (1-\gamma_T) (1 - \frac{t}{T})^3$. We adaptively increase the tuning parameters in the pruning stage but restrict them to a specific limit $\Delta_t$ at each training step $t$.
% \hanna{you can connect these to the variables you defined in the method section. is it one of the delta's? } 
Towards better training stability in {\lmabbr} pruning, we gradually decrease the pruning masks of pruned blocks by $\alpha < 1$ instead of instantly setting them from ones to zeros. We also use the exponential moving-averaged salience in~\citep{zhang2023adaptive} when calculating the salience score during fine-tuning.
% For instance, when tuning RoBERTa models, we only tune 0.53\% to 2.13\% of the model parameters. \hanna{why for example? how about other llms?} \todo{maybe moving this into the tables?}


\section{Block salience calculation and correlations} \label{appendix:salience-unify}
As addressed in \cref{sec:apt}, we use the compressed weight-gradient production as the salience metric for identifying the tuning and pruning parameter blocks in {\lmabbr}s. Previous works~\citep{sanh_movement_2020} use salience score defined as the magnitude of the parameters' weight-gradient production, where given a linear layer $H = WX$ (we omit the bias term here for simplicity) in model parameters $\Theta$ trained on the objective $\mathcal{L}$, the salience scoring function $S$ is defined as:

\begin{equation} \label{eq:sensitivity}
    \begin{split}
        S(W_{i,j}) &= \sum_{(x, y) \in \mathcal{D}} s(W_{i,j}, x, y) \\
        &= \sum_{(x, y) \in \mathcal{D}}|\frac{\partial \mathcal{L}(x, y | \Theta)}{\partial W_{i,j}} \cdot W_{i,j}| \\
        S(W_{:,j}) &= \sum_{(x, y) \in \mathcal{D}}\sum_{i}|\frac{\partial \mathcal{L}(x, y | \Theta)}{\partial W_{i,j}} \cdot W_{i,j}| \\
        &= \sum_{(x, y) \in \mathcal{D}} (\sum_{i}|\frac{\partial \mathcal{L}(x, y | \Theta)}{\partial X_{j, i}} \cdot X_{j, i}|)
    \end{split}
\end{equation}

where $x, y$ are the inputs and labels sampled from the training batch $\mathcal{D}$. $S(W_{i,j})$ denotes the unstructured, sparse parameter's salience, and $S(W_{:, j})$ denotes the salience score of a block in the weight $W$ (for example, rows, columns, attention heads, etc.).

When applying this equation to {\ourarchabbr} layers as defined in \cref{eq:elastic-lora}, there are three different consistent dimensions, namely input dimension $j$, output dimension $i$, and tuning rank dimension $k$. Therefore, the combined salience (including tuning low-rank weights and the frozen weight) of the parameter block shall be calculated as follows:
\begin{equation} \label{eq:appendix-elastic-salience}
    \begin{split}
        S(H, i) &= \sum_l \frac{\partial \mathcal{L}(x, y | \Theta)}{\partial H(X)_{i,l}} \cdot H(X)_{i,l} \\
                &= \sum_p \frac{\partial \mathcal{L}(x, y | \Theta)}{\partial W_{i,p}} \cdot W_{i,p} \\
                &+ s\cdot \sum_q \frac{\partial \mathcal{L}(x, y | \Theta)}{\partial {W_B}_{i,q}} \cdot {W_B}_{i,q} \\
        S(H, j) &= \sum_l \frac{\partial \mathcal{L}(x, y | \Theta)}{\partial X_{j,l}} \cdot X_{j,l} \\
                &= \sum_p \frac{\partial \mathcal{L}(x, y | \Theta)}{\partial W_{p,j}} \cdot W_{p,j} \\
                &+ s\cdot \sum_q \frac{\partial \mathcal{L}(x, y | \Theta)}{\partial {W_A}_{q,j}} \cdot {W_A}_{q,j} \\
        S(H, k) &= s\cdot \sum_l \frac{\partial \mathcal{L}(x, y | \Theta)}{\partial {W_A}_{k,l}} \cdot {W_A}_{k,l} \\
                &= s\cdot \sum_l \frac{\partial \mathcal{L}(x, y | \Theta)}{\partial {W_B}_{l,k}} \cdot {W_B}_{l,k} \\
    \end{split}
\end{equation}
Therefore, we can notice that the real block-wise salience of the LoRA layer shall be the sum of the block-wise frozen weight salience and the corresponding tuning weight. Hence, the existing work~\citep{zhang2023pruning} that only uses the tuning block salience as layer salience leads to sub-optimal pruning results. Meanwhile, we shall also notice the correlation between the input-, output-dimension, and tuning rank dimensions, which are the summation of the weight-gradient production of parameters on different dimensions. 

\section{Adaptive Pruning and Tuning Details} \label{appendix:binary-search}

\begin{algorithm*}[t!]
 	\caption{Adaptive Pruning and Tuning} 
 	\label{alg:epa}
 	\begin{algorithmic}[1]
 		\STATE {{\bfseries Input:} Model $f$; Training dataset $ \mathcal{D} $; total training steps $ T $; Adjustment step set $\mathcal{T}$; Training target $\mathcal{L}$; Initial parameters and masks $\Theta_0, M_0$, training memory budget $\Delta$; Parameter number constraint $\gamma$; Hyperparameters $\alpha\,\beta$.}  
 		\FOR{$ t = 1, \dots, T $}  
            \STATE Forward pass: \( L \leftarrow \mathcal{L}(f(\Theta_t, D_t)) \)
            \STATE Cache the batch-sequence summed hidden states: $\widetilde{H} \leftarrow \sum_{i, j} (|H|)_{ij}$
 		\STATE Backward pass: \( \nabla_{\Theta_t} L \leftarrow \frac{\partial \mathcal{L}(f(\Theta_t, D_t))}{\partial \Theta_t} \)
            \STATE Calculate approximated salience: $\widetilde{S}(m_i) \leftarrow \widetilde{H} \cdot \sum_{i, j} (|\nabla_{H} L|)_{ij}$
            \STATE Update global scores: $\overline{S}^{(t)}(m) \leftarrow \beta \overline{S}^{(t-1)}(m) + (1-\beta) \widetilde{S}(m)$;
 		\STATE Select blocks: $M_1, M_0 \leftarrow$ Binary search against constraint \cref{eq:parameter-constraint}, with scores $\overline{S}^{(t)}(m)$;
            \STATE Update masks: $M^{(t)}_1 \leftarrow min(1, M^{(t-1)}_1 + \alpha)$, $M^{(t)}_0 \leftarrow max(0, M^{(t-1)}_0 - \alpha)$;
            \STATE Update parameters: $\Theta_{t+1} \leftarrow \Theta_t - \alpha \nabla_{\Theta_t} L$
 		\ENDFOR
 		\STATE \textbf{Output:}  { Parameters and masks $ \Theta^{(T)}, M^{(T)}$.} 
	\end{algorithmic}
\end{algorithm*}

We show the detailed algorithm description of our Lightweight Parameter Adjustment as described in \cref{sec:apt} in \cref{alg:epa}. For the details of the algorithm, we first sort all blocks by the salience density, defined as the block salience divided by the number of parameters in the block. For instance, given a RoBERTa-base model with the hidden dimension $d_m = 768$, the number of transformer layers $n_L = 12$, the number of attention heads $n_h = 12$, and the number of FFN neurons $n_f = 3072$, we have:
\begin{align}
    \mathcal{C}_{\text{head}} &= 4 \times d_m \times d_m / n_h = 196608\\
    \mathcal{C}_{\text{neuron}} &= 2 \times d_m = 1536\\
    \mathcal{C}_{\text{dimension}} &= n_L \times (4 d_m + 2 n_f) = 110592
\end{align}
We also omit the bias term for density calculation since it takes up less than 1\% of {\lmabbr}'s parameters. Since the number of heads, neurons, and hidden dimensions is ever-changing during pruning, we re-calculate the density after executing each parameter size change. Meanwhile, for T5 and LLaMA-like models, the FFN layers are gated, consisting of up-, gate-, and down-projection linear layers. Therefore, the number of layers in FFN shall be three instead of two in these {\lmabbr}s. Furthermore, for encoder-decoder {\lmabbr}s like T5, the cross-attention layers in the decoder shall also be counted.

After sorting the blocks by salience density, as {\lmabbr}'s parameter size monotonically increases with the number of MHA heads, FFN neurons, and hidden dimensions, we conduct a binary search algorithm to identify the blocks shall be retained as {\lmabbr}'s parameter size monotonically increases with the number of MHA heads, FFN neurons, and hidden dimensions. Specifically, given a sorted list of $N$ blocks $B = \{b_1,b_2,...,b_N\}$ and function $f$ for identifying the block's category where
\begin{equation}
    f(b_i) = 
    \begin{cases} 
    0 & \text{if } b_i \text{ is a head} \\
    1 & \text{if } b_i \text{ is a neuron} \\
    2 & \text{if } b_i \text{ is a dimension} \\
    \end{cases}
\end{equation}
given any index $i$, we can calculate the parameter number of the {\lmabbr} consisting of the top-$i$ blocks by:
\begin{equation}
\begin{split}
    \mathcal{C}_{\text{top-}i} &= (4 d_h' \cdot n_h' + 2 n_f') \cdot d_m' \\
    n_h' &= \sum_{j=0}^{i-1} \delta(0, f(b_j)) \\
    n_f' &= \sum_{j=0}^{i-1} \delta(1, f(b_j)) \\
    d_m' &= \sum_{j=0}^{i-1} \delta(2, f(b_j)) \\
\end{split}
\end{equation}
where $\delta(i, j)$ is the Kronecker delta function that valued 1 if $i=j$ and otherwise 0. Hence, we can use binary search to get the top-$i$ salient blocks, which shall be retained given a parameter constraint, and the rest of the block shall be pruned. In our implementation, for training stability, we do not set the pruned blocks' corresponding masks to 0 directly but gradually decrease their values by $\alpha = 0.01$.

% \begin{equation}
%     \begin{split} \label{eq:parameter-constraint}
%         \mathcal{C}(\Theta, \mathcal{M}) &= \sum_{k=1}^n \mathcal{C}(\theta^k, M^k) = \sum_{k=1}^{n_f} \mathcal{C}(\theta^k_f, M^k_f) + \sum_{k=1}^{n_t} \mathcal{C}(\theta^k_t, M^k_t) \\
%         &\approx \sum_{k_f} (\sum m^{k_f}_p \cdot \sum m^{k_f}_h) + \sum_{k_t} ((\sum m^{k_t}_p + \sum m^{k_t}_h) \cdot \sum m^{k_t}_r \\
%         &\approx \sum_{k_f} (\sum m^{k_f}_p \cdot \sum m^{k_f}_h)
%     \end{split}
% \end{equation}

% Please add the following required packages to your document preamble:
% \usepackage{booktabs}
% \usepackage{multirow}
\begin{table*}[htbp]
\centering
\resizebox{1.0\textwidth}{!}{
\begin{tabular}{@{}rlrrrrrrrrr@{}}
\toprule
Density               & Method  & MNLI          & QQP           & QNLI          & SST2          & CoLA          & STS-B         & MRPC          & RTE           & GLUE Avg.     \\ \midrule
\multirow{5}{*}{50\%} & MaP & \textbf{83.6} & {\ul 87.8}    & \textbf{91.5} & 91.0          & \textbf{60.1} & \textbf{89.8} & 90.7          & 67.2          & {\ul 82.7}    \\
                      & MvP & 82.3          & 87.3          & {\ul 90.8}    & 90.8          & 57.7          & {\ul 89.4}    & {\ul 91.1}    & 67.2          & 82.1          \\
                      & PST & 81.0            & 85.8          & 89.8          & {\ul 91.3}    & 57.6          & 84.6          & 90.7          & 67.9          & 81.0            \\
                      & LRP & 82.4          & 87.2          & 89.6          & 90.9          & 54.1          & 88.7          & 89.8          & {\ul 69.3}    & 82.2          \\ \cmidrule(l){2-11} 
                      & APT & {\ul 82.8}    & \textbf{90.1} & 90.1          & \textbf{92.7} & {\ul 59.6}    & 88.3          & \textbf{91.8} & \textbf{70.4} & \textbf{83.2} \\ \midrule
\multirow{5}{*}{10\%} & MaP & 78.2          & 83.2          & 84.1          & 85.4          & 27.9          & 82.3          & 80.5          & 50.1          & 71.4          \\
                      & MvP & \textbf{80.1} & 84.4          & \textbf{87.2} & 87.2          & 28.6          & {\ul 84.3}    & 84.1          & 57.6          & 74.2          \\
                      & PST & {\ul 79.6}    & {\ul 86.1}    & {\ul 86.6}    & 89.0            & \textbf{38.0}   & 81.3          & 83.6          & {\ul 63.2}    & {\ul 75.9}    \\
                      & LRP & 79.4          & 86.0            & 85.3          & {\ul 89.1}    & {\ul 35.6}    & 83.3          & {\ul 84.4}    & 62.8          & 75.7          \\ \cmidrule(l){2-11} 
                      & APT & 78.8          & \textbf{89.4} & 85.5          & \textbf{90.0}   & 30.9          & \textbf{86.3} & \textbf{88.2} & \textbf{65.3} & \textbf{76.8} \\ \bottomrule
\end{tabular}
}
\caption{Comparison of {\ourmethod} to existing unstructured pruning baseline with using PEFT in conjunction. The best results are \textbf{bold} while the second-best ones are \ul{underlined}.}
\label{tab:bert-additional-result}
\end{table*}

% Please add the following required packages to your document preamble:
% \usepackage{multirow}
\begin{table*}[htbp]
\centering
\begin{tabular}{ll|rrrrrrrr}
\hline
Sparsity              & Method       & MNLI & QQP  & QNLI & SST2 & CoLA & MRPC & RTE  & GLUE Avg. \\ \hline
\multirow{2}{*}{0\%}  & FT           & 87.6 & 91.9 & 92.8 & 95.2 & 91.2 & 90.2 & 78.7 & 89.7      \\
                      & LoRA         & 87.5 & 90.8 & 93.3 & 95.0 & 63.4 & 89.7 & 72.1 & 84.5      \\ \hline
\multirow{2}{*}{40\%} & LoRA+Distill & 84.2 & 88.3 & 90.1 & 91.9 & 49.9 & 86.8 & 68.6 & 80.0      \\ 
                      & APT          & 86.4 & 90.9 & 92.3 & 94.5 & 56.5 & 92.3 & 74.4 & 83.9      \\ \hline
\end{tabular}
\caption{Detailed results of RoBERTa pruning with {\ourmethod} compared to the LoRA+Distill baseline. We ignore the evaluation results of the STS-B task since it cannot be successfully reproduced with CoFi (the distillation backbone).}
\label{tab:roberta-detailed-results}
\end{table*}

\section{Additional Baseline Comparisons} \label{sec:appendix-additional-exp}
In this section, we further compare {\ourmethod} to existing parameter-efficient pruning methods, such as PST and LRP. In the meantime, we also show detailed results of {\ourmethod} pruning compared to the LoRA+Distill baseline with more tasks in the GLUE benchmark and LLaMA-2 13B model pruning results.

\subsection{Comparison to PST and LRP}
We compare {\ourmethod} with the state-of-the-art joint use of unstructured pruning~\citep{ijcai2022p586} and structured pruning~\citep{zhang2023pruning} with PEFT on $\text{BERT}_{\text{base}}$ model, showing in \cref{tab:bert-additional-result}. We can see that {\ourmethod} outperforms existing baselines in both 50\% and 10\% pruning density settings with a notable margin. The performance gain is credited to our more accurate pruning strategy considering frozen and tuning parameters. At the same time, our efficient self-distillation technique used in conjunction with salient parameters added in training also boosts performance recovery.

\subsection{Further Comparison to LoRA+Distill}
We show the detailed comparison between {\ourmethod} and the LoRA+Distill baseline in \cref{tab:roberta-detailed-results}. {\ourmethod} reaches superior task performance compared to the baseline in all seven GLUE tasks listed in the table, with on average 93.5\% fine-tuned {\lmabbr} performance maintained, notably outperforming the joint use of LoRA and knowledge distillation. In particular, the results of STS-B cannot be reproduced when conducting CoFi distillation with LoRA parameters tuned only, so we exclude the comparison on STS-B. Among the other seven tasks in the GLUE benchmark, we find that tasks with relatively smaller dataset sizes, namely CoLA, MRPC, and RTE, reach superior performance gain when using {\ourmethod}. We conclude that this is because, compared to simple fine-tuning, knowledge distillation with salient parameters added in training is more robust and not prone to overfitting the training data.

\subsection{LLaMA-2 13B Pruning Results}
% Please add the following required packages to your document preamble:
% \usepackage{booktabs}
% \usepackage{multirow}
\begin{table}[htbp]
\centering
\begin{tabular}{@{}l|lrrrr@{}}
\toprule
Method                            & ARC           & HellaSwag     & MMLU          & TruthfulQA    & Avg.          \\ \midrule
LLaMA2 7B                              & 53.1          & 77.7          & 43.8          & 39.0          & 53.4          \\ \midrule
LoRA                              & 55.6          & 79.3          & 46.9          & 49.9          & 57.9          \\ \midrule
% LoRA + $\text{MT}$                & 42.1          & 49.1          & 25.6          & 43.5          & 40.1          \\
LoRA+Prune & \textbf{46.8} & 65.2          & 23.9          & 46.2          & 45.5          \\
LLMPruner                         & 39.2          & 67.0          & 24.9          & 40.6          & 42.9          \\
APT                               & 45.4          & \textbf{71.1} & \textbf{36.9} & \textbf{46.6} & \textbf{50.0} \\ \midrule
LLaMA2 13B                       & 59.4          & 82.1          & 55.8          & 37.4          & 58.7          \\ \midrule
LoRA                              & 60.8          & 82.8          & 56.0          & 46.5          & 61.5          \\ \midrule
% LoRA + $\text{MT}$                & 51.2          & 74.0          & 44.7          & 39.9          & 52.4          \\
LoRA+Prune & \textbf{56.4} & \textbf{79.1} & 50.7          & 42.1          & \textbf{57.1} \\
LLMPruner                         & 46.8          & 74.0          & 24.7          & 34.8          & 45.1          \\
APT                               & 49.5          & 75.8          & \textbf{52.5} & \textbf{44.7} & 55.6          \\ \bottomrule
\end{tabular}
\caption{LLaMA2 7B and 13B 30\% sparsity pruning results with GPT4-generated Alpaca dataset, evaluated on the Open LLM leaderboard few-shot tasks.}
\label{tab:llama-results}
\vspace{-10pt}
\end{table}

As shown in \cref{tab:llama-results}, when pruning LLaMA-2 13B models, {\ourmethod} maintains 90.0\% performance of the unpruned LoRA-tuned baseline. Compared to the pruning result on 7B models that maintain 86.4\% dense model performance, better accuracies can be recovered in larger models (13B). At the same time, under the same pre-tuning pruning settings, {\ourmethod} performs better than the LLMPruner baseline on all four evaluation tasks, indicating the effectiveness of considering outlier parameters in large {\lmabbr} pruning. Nonetheless, the LoRA+Prune baseline reaches slightly better results than {\ourmethod} when pruning 13B models, illustrating that there is still room for improving pre-tuning pruning methods in future works. More specifically, among the four tasks we use for evaluating large {\lmabbr}s, TruthfulQA benefits the most from Alpaca fine-tuning. We can see that {\ourmethod} reaches superior results on TruthfulQA than existing baselines regardless of model size. The {\lmabbr}'s capabilities on ARC and HellaSawg downgrade the most when pruning large {\lmabbr} before fine-tuning, implying possibilities of catastrophic forgetting in this paradigm.



% \section{Detailed Efficiency Evaluation}
% We show the detailed efficiency evaluation results of \cref{fig:perf-efficiency-tradeoff} in \cref{tab:roberta-base-sst2} with 40\% density pruning of the $\text{RoBERTa}_{\text{base}}$ model on SST-2 dataset, where we can notice that {\ourmethod} can reach a certain accuracy faster then LoRA and results in substantial inference speedup and memory reduction, while the training memory cost overhead is also minimal compared to LoRA.

% The detailed efficiency comparisons of RoBERTa, T5, and LLaMA models are shown in Fig.~\ref{fig:perf-efficiency-tradeoff}.

% \begin{figure}[htbp]
%     \begin{minipage}{0.5\textwidth}
%         \begin{figure}[H]
%             \centering
%             \includegraphics[width=\textwidth]{figures/roberta_base_sst2_radar.pdf}
%         \end{figure}
%     \end{minipage}
%     \caption{Radar map for performance, training, and inference efficiency on RoBERTa SST2.}
%     \label{fig:radar}
% \end{figure}

% \todo{Radar map needs font refinement}

% We use radar charts to thoroughly compare the training-inference efficiency and the model's performance with different tuning-pruning strategies, where we set the score of the best and worst results for each metric to 1 and 0 while applying linear transformation for the results in between. As shown in \cref{fig:radar}, instead of focusing on only one aspect (either efficiency or performance) of fine-tuning, {\ourmethod} achieves the best tradeoff for all metrics indicating training-inference time and memory consumption.

% % Please add the following required packages to your document preamble:
% \usepackage{booktabs}
\begin{table*}[htbp]
\begin{tabular}{@{}lrrrrrrrr@{}}
\toprule
Method    & Target Density & Init Density & T. M.   & ARC   & HellaSWAG & MMLU  & TruthfulQA & Avg.    \\ \midrule
None      & 100\%           & 100\%         & 100\%    & 59.39 & 82.13     & 55.77 & 37.38      & 58.6675 \\
LoRA      & 100\%           & 100\%         & 100\%    & 60.84 & 82.82     & 55.95 & 46.5       & 61.5275 \\ \midrule
LoRA + $\text{MT}_{\text{train}}$ & 70\%            & 70\%          & 100\%    & 56.4  & 79.08     & 50.71 & 42.07      & 57.065  \\
LLMPruner & 70\%            & 70\%          &         &       &           &       &            &         \\
APT       & 70\%            & 100\%         & 100.50\% & 45.9  &           &       & 47.37      &         \\
APT       & 70\%            & 85\%          & 86.24\%  & 47.95 & 75.61     & 50.28 & 46.33      & 55.0425 \\
APT       & 70\%            & 70\%          & 75.82\%  & 49.49 & 75.84     & 52.46 & 44.67      & 55.615  \\ \midrule
LoRA + $\text{MT}_{\text{train}}$ & 50\%            & 50\%          & 100\%    & 38.4  & 60.55     & 42.92 & 46.51      & 47.095  \\
LLMPruner & 50\%            & 50\%          &         &       &           &       &            &         \\
APT       & 50\%            & 100\%         &         &       &           &       &            &         \\
APT       & 50\%            & 75\%          & 78.03\%  & 36.86 & 57.15     &       & 41.54      &         \\
APT       & 50\%            & 50\%          &         & 36.35 & 55        & 30.71 & 45.44      & 41.875  \\ \bottomrule
\end{tabular}
\caption{Main results of {\ourmethod} compared to structured pruning baselines on LLaMA 2 13B model.}
\label{tab:llama-2-13b-main-result}
\end{table*}

% % Please add the following required packages to your document preamble:
% \usepackage{booktabs}
\begin{table*}[htbp]
\centering
\begin{tabular}{@{}lr|rrrrrr@{}}
\toprule
Method                         & Density & SST2 & 97\% TTA     & 99\% TTA     & T. M. & Inf. $\nearrow$ & Inf. Mem. \\ \midrule
FT                             & 100\%   & 94.8 & 1.00$\times$ & 1.00$\times$ & 100\%           & 1.00$\times$ & 100.0\%  \\
LoRA                           & 100\%    & 95.1 & 21.4$\times$ & 8.39$\times$ & 60.5\%          & 1.00$\times$ & 100.0\%  \\ \midrule
FT + $\text{MT}_{\text{train}}$              & 40\%    & 93.5 & 6.80$\times$ & -            & 100\%           & 2.67$\times$ & 76.4\%   \\
LoRA + $\text{MT}_{\text{train}}$            & 40\%    & 93.0   & 51.3$\times$ & -            & 60.5\%          & 2.63$\times$ & 75.1\%   \\
% LoRA* + $\text{MT}_{\text{train}}$ & 40\%    & 93.2 & 51.3$\times$ & -            & 60.5\%          & 2.57$\times$ & 74.2\%   \\
FT + $\text{MT}_{\text{distill}}$              & 40\%    & 94.3 & 9.40$\times$ & 6.17$\times$ & 100\%           & 2.60$\times$ & 76.4\%   \\
LoRA + $\text{MT}_{\text{distill}}$            & 40\%    & 92.3 & 65.3$\times$ & -            & 70.5\%          & 2.60$\times$ & 75.1\%   \\
% LoRA* + $\text{MT}_{\text{distill}}$ & 40\%    & 92.8 & 65.3$\times$ & -            & 70.5\%          & 2.60$\times$ & 75.1\%   \\
FT + CoFi                      & 40\%    & 94.5 & 15.0$\times$ & -            & 168.5\%         & 2.59$\times$ & 79.2\%   \\
LoRA + CoFi                    & 40\%    & 91.9 & 65.3$\times$ & -            & 141.4\%         & 2.54$\times$ & 82.3\%   \\ \midrule
{\ourmethod}                  & 40\%    & 94.5 & 5.92$\times$ & 4.13$\times$ & 62.7\%          & 2.71$\times$ & 79.5\%   \\ \bottomrule
\end{tabular}
\caption{RoBERTa-base SST2 performance and efficiency results. LoRA* denotes the setting where we use the model tuned with LoRA after the same amount of epochs when fine-tuning converges as the trained/teacher model. T.M. denotes the relative training peak memory to full fine-tuning. Inf. $\nearrow$ and Inf. Mem. represent the relative inference speedup and memory usage compared to full fine-tuning.}
\label{tab:roberta-base-sst2}
\end{table*}

\section{Efficiency and Performance Tradeoff Analysis}

\begin{figure*}[t!]
    \centering
    \includegraphics[width=\textwidth]{figures/plot_tradeoff_scatter.pdf}
    \caption{The performance-efficiency tradeoff of {\ourmethod} compared to baseline methods. All metrics are normalized using LoRA tuning w/o pruning as the baseline. The circular dots with vertical axes on the left indicate training speed v.s. performance, with their sizes denoting the peak training memory usage. The squared dots with axes on the right indicate inference speedup v.s. performance, with sizes denoting inference memory usage.}
    \label{fig:perf-efficiency-tradeoff}
    \vspace{-10pt}
\end{figure*}

We use \cref{fig:perf-efficiency-tradeoff} to clearly show the {\lmabbr}s' end-task performance and efficiency tradeoffs between different tuning, pruning, and distillation baselines. We add several extra baselines to conduct more detailed comparisons between {\ourmethod} with existing PEFT, pruning, and distillation methods: 

\noindent \textbf{LoRA+Prune w/distill}: we first use LoRA to fully converge a model on the task dataset, and then use Mask-Tuning~\citep{kwon_fast_2022} to prune the {\lmabbr}. Afterward, we utilize the converged model before pruning as the teacher model and distill its knowledge to the pruned student model with static knowledge distillation objectives.

\noindent \textbf{LoRA+Prune w/o retrain}: we use Mask-Tuning to prune a LoRA-tuned converged model but do not conduct any retraining to recover the pruned models' performance. Therefore, the {\lmabbr}'s training time will be reduced, yet its performance is lower than the LoRA+Prune baseline.

With the same target sparsity in RoBERTa and LLaMA pruning setups, {\ourmethod} achieves on-par end-task performance with full fine-tuning and LoRA tuning baselines. Meanwhile, {\ourmethod}-tuned models reach similar or even better inference time and memory efficiency than existing baselines. {\ourmethod}-pruned T5 {\lmabbr}s' inference efficiency is slightly worse because more decoder parameters (with less computations happening) are pruned than the baselines. Moreover, when pruning RoBERTa and T5 models, {\ourmethod} achieves faster training time than all pruning and distillation baselines. Specifically, the training speed of {\ourmethod} in RoBERTa models is even higher than LoRA tuning without pruning. In LLaMA model pruning, {\ourmethod} costs significantly less training memory than both LLMPruner and LoRA+Prune baselines.

\section{Pruning Sparsity Analysis} \label{appendix:sparsity}

\begin{figure}[ht!]
  \centering
  % \subfloat[Task performance v.s. relative inference efficiency on RoBERTa, T5, and LLaMA-2 7B models with {\ourmethod} and baselines.]{
  %   \includegraphics[width=0.4\columnwidth]{figures/pruning_tradeoff.pdf}
  %   \label{fig:prune-tradeoff}
  % }
  % \hfill
  \begin{minipage}[b]{\textwidth}
    \subfloat[Comparison of different initial ranks of LoRA layers pruning with {\ourmethod} on RoBERTa with SST2 task accuracy, relative training peak memory and speed to 97\% fine-tuning accuracy to the fine-tuning model.]{
    \includegraphics[width=0.45\columnwidth]{figures/tuning_tradeoff.pdf}
    \label{fig:rank-tradeoff}
    }
    \hfill
    \subfloat[Training initial sparsity trade-off with 30\% target sparsity model's relative performances to the LoRA-tuned LLaMA2-7B and 13B models.]{
        \includegraphics[width=0.45\columnwidth]{figures/llama_init_density.pdf}
        \label{fig:llama-init-tradeoff}
    }
  \end{minipage}
  \caption{Detailed analysis in {\ourmethod} with different initial, target sparsities, and adaptive tuning schedules.}
  \label{fig:figureTableCombo}
\end{figure}

% \begin{figure}[t!]

%     % \vspace{-10pt}
% \end{figure}

We further show the task performance changing trajectory with different pruning sparsities in \cref{fig:prune-tradeoff}. {\ourmethod} achieves superior inference speedup and less inference memory consumption than baselines targeting the same task performance. Compared to the LoRA+Prune baseline, when pruning RoBERTa models targeting similar task accuracy, {\ourmethod} gains 21.8\% more inference speedup and 7\% more memory reduction. For T5 model pruning with 97\% dense model performance maintained, {\ourmethod} results in 62.7\% more inference speedup with 24.8\% more inference memory reduced compared to the LoRA+Prune baseline.
When pruning large LLaMA2-7B models, {\ourmethod} prunes gets 6.7\% more speedup and 9.2\% more inference memory reduction than the LoRA+Prune baseline, with about 85\% dense model performance maintained.

\section{Distillation Strategy Comparison} \label{appendix:distillation}

% Please add the following required packages to your document preamble:
% \usepackage{booktabs}
% \begin{table}[htbp]
\begin{table}[htbp]
\centering
\begin{tabular}{lr|rr}
\hline
                        & SST2 & Train. Speed($\Uparrow$)      & Train. Mem.($\Downarrow$)   \\ \hline
\ourmethod                             & 94.5 & 16.9\% & 70.1\% \\
\tableindent w/o $\mathcal{L}_{layer}$ & 93.7 & 17.4\%  & 69.8\% \\
\tableindent w/o self-distillation          & 92.9 & 20.7\%  & 69.2\% \\ \hline
FT teacher         &   94.3  &  7.9\% & 111.8\%  \\
LoRA teacher       &   93.7 &  1.7\%   &  96.1\%\\ \hline
\end{tabular}
\caption{Ablation study of distillation strategies and comparison to non-efficient distillation techniques. The training speed and memory are relative metrics compared to fine-tuning the dense model.}
\label{tab:ablate-distillation}
\vspace{-10pt}
\end{table}

We show the further analysis in \cref{tab:ablate-distillation} to compare the self-distillation technique we use in {\ourmethod} and traditional knowledge distillation methods. When ablating the dynamic layer mapping strategy in our self-distillation approach, the {\lmabbr} performance decreased by 0.8\% with similar training time and memory consumption. When training without distillation objectives (w/o self-distillation), the {\lmabbr} performance drops by 1.7\%. Nonetheless, the training is slightly faster with less memory costs. These results present that using distillation objectives for better {\lmabbr} task performance will sacrifice training efficiency as a tradeoff.
Furthermore, we also demonstrate the comparisons with existing static knowledge distillation strategies, using the converged full-parameter fine-tuned {\lmabbr} (FT teacher) and LoRA-tuned {\lmabbr} (LoRA teacher) as the teacher model. We calculate the time consumption for both teacher and student training when using these distillation baselines. As shown in \cref{tab:ablate-distillation}, using fully fine-tuned models as the teacher will incur more memory cost than dense model fine-tuning, while {\ourmethod} only consumes 70\%. In the meantime, the training convergence speed of {\ourmethod} training is two times faster than the traditional knowledge distillation method with a fine-tuned teacher. Furthermore, using a LoRA-tuned model as the teacher will result in extremely slow training speed. In addition, simply tuning the LoRA layers with knowledge distillation objectives doesn't help reduce the training memory consumption, as the memory consumption is still 96.1\% than full fine-tuning.

\section{Adaptive Pruning and Tuning Analysis} \label{sec:analysis}
% \noindent \textbf{Does bigger bottleneck dimension help?} \label{sec:discuss-r}

% \begin{figure}[htbp]
%     \centering
%     \includegraphics[width=0.5\textwidth]{figures/lora_r_corr.pdf}
%     \caption{The correlation between the starting LoRA rank dimension with training time speedup, memory reduction, and overall accuracy on MNLI dataset with $\text{RoBERTa}_{\text{base}}$ model.}
%     \label{fig:corr-lora-r}
% \end{figure}
% \noindent \textbf{{\ourmethod} recovers task performance under different inference efficiency settings.} Shown in \cref{fig:prune-tradeoff}, {\ourmethod} maintains more than 91.2\% accuracy compared to fine-tuning under 95\% sparsity, gaining 6.8$\times$ speedup while reducing the inference memory consumption to 42.7\%. Pruning becomes challenging when the sparsity is more than 90\% for small models. When pruning billion-level LLMs such as the LLaMA-2 model, as depicted in \cref{tab:llama-results}, {\ourmethod} maintains 86.4\% performance for 7B model and 90.4\% performance for 13B model when 30\% parameters are pruned. In the meantime, comparing \cref{fig:prune-tradeoff} and \cref{fig:llama-tradeoff}, the task performance drop brought by pruning is more significant for LLaMA models. The result implies that larger models (13B v.s. 7B) could have more redundant parameters. However, pruning them under instruction-tuning is more challenging than single-task fine-tuning (LLaMA v.s. RoBERTa).

% \todo{Better clarify: adding more parameters in fine-tuning}
\noindent \textbf{Effects of adaptive tuning strategies on end-task performance and training efficiency.}
As the trajectories shown in \cref{fig:rank-tradeoff}, simply enlarging the initial tuning parameter number in {\ourmethod} will not improve or even hurt the model's final performance. Moreover, the training memory consumption grows even higher than fine-tuning when the tuning layer ranks become extremely large (initial ranks set as 256). Therefore, this result proves that adding tuning parameters according to layer salience is better than uniformly increasing them before tuning.

% \todo{pruning before fine-tuning xxx}
\noindent \textbf{Effects of early pruning on task accuracy and training memory in LLaMA pruning.}
\cref{fig:llama-init-tradeoff} shows the effect of the initial density on LLaMA models' task performance under the 30\% sparsity pruning setting. We find that densely-trained models only perform better in TruthfulQA with fewer parameters pruned before tuning. The accuracy reaches 48.6 and 47.4 when not pruning before tuning, compared to 46.6 and 44.7 when directly pruning to the target sparsity for both 7B and 13B models. Training the {\lmabbr} densely harms the model performance while costing extra memory for all other tasks. These results demonstrate that pruning during training hurts large LM performance under distillation-free settings, and we hypothesize this is due to the training instability issue when parameters are set to zeros during fine-tuning.
% On the other hand, it also opens up the possibility that with decent structured pruning strategies, one can discard irrelevant parameters in large {\lmabbr} before training for superior training efficiency.

\section{Absolute Efficiency Metrics} \label{sec:raw-efficiency-results}
We report the raw efficiency evaluation results in \cref{tab:raw-efficiency} and \cref{tab:raw-llama-efficiency}, including training and inference time and memory consumption. The training times are measured in seconds, and the inference times are measured in milliseconds. All memory footprints are measured in MB. We report the time-to-accuracy for RoBERTa and T5 model training to measure the training time. For LLaMA model training, we measure the training time per epoch to represent training time consumption. 

% Please add the following required packages to your document preamble:
% \usepackage{booktabs}
% \usepackage{multirow}
\begin{table}[htbp]
\centering
\begin{tabular}{@{}llr|rrrr@{}}
\toprule
Model                                           & Method             & Sparsity & 97\% TTA (s) & Train Mem. (MB) & Inf. Time (ms) & Inf. Mem (MB) \\ \midrule
\multirow{6}{*}{$\text{RoBERTa}_{\text{base}}$} & FT                 & 0\%      & 127          & 2,696           & 220.8          & 1,157         \\
                                                & LoRA               & 0\%      & 2,714        & 1,630           & 181.8          & 1,157         \\ \cmidrule(l){2-7} 
                                                & LoRA+Prune         & 60\%     & 6,513        & 1,630           & 84.0           & 869           \\
                                                & Prune+Distill      & 60\%     & 1,899        & 4,544           & 85.2           & 917           \\
                                                & LoRA+Prune+Distill & 60\%     & 8,299        & 3,813           & 87.0           & 952           \\ \cmidrule(l){2-7} 
                                                & APT                & 60\%     & 752          & 1,890           & 91.3           & 904           \\ \midrule
\multirow{4}{*}{$\text{T5}_{\text{base}}$}      & FT                 & 0\%      & 366          & 7,217           & 248.1          & 2,347         \\
                                                & LoRA               & 0\%      & 935          & 4,476           & 254.2          & 2,347         \\ \cmidrule(l){2-7} 
                                                & LoRA+Prune         & 60\%     & 14,417       & 4,476           & 116.8          & 1,724         \\ \cmidrule(l){2-7} 
                                                & APT                & 60\%     & 1,774        & 5,332           & 185.0          & 1,913         \\ \bottomrule
\end{tabular}
\caption{Raw efficiency metrics, including time to accuracy, training peak memory, inference time and memory footprints, when using different methods to fine-tune $\text{RoBERTa}_{\text{base}}$ and $\text{T5}_{\text{base}}$models on SST2.}
\label{tab:raw-efficiency}
\end{table}

% Please add the following required packages to your document preamble:
% \usepackage{booktabs}
\begin{table}[htbp]
\centering
\begin{tabular}{@{}l|rrrr@{}}
\toprule
Method          & Train Time (s) & Train Mem. (MB) & Inf. Time (ms) & Inf. Mem (MB) \\ \midrule
LoRA            & 980            & 32,185          & 2457.5         & 45,311        \\
LoRA+MT         & 980            & 32,185          & 2127.5         & 31,207        \\
LoRA+MT+retrain & 1,773          & 32,185          & 2127.5         & 31,207        \\
LLMPruner       & 852            & 23,425          & 2140.6         & 33,625        \\ \midrule
APT             & 1,039          & 24,408          & 2099.7         & 30,469        \\ \bottomrule
\end{tabular}
\caption{Raw efficiency metrics, including time to accuracy, training peak memory, inference time, and memory footprints, when using different methods to fine-tune LLaMA2 7B models on Alpaca.}
\label{tab:raw-llama-efficiency}
\end{table}
\end{document}
