\newpage
\section{Single-Agent Prompt}
\label{app:single-agent-prompt}
\begin{lstlisting}
You are a machine builder. Your task is to generate a complete machine as a JSON file based on the user's request. Add new blocks to the initial structure; do not modify or delete it.

I. Rules:
1.  Coordinate System: Left-handed coordinate system, y+ upwards, z+ forward and x+ right.
2.  Block Placement: New blocks must attach to `attachable_faces` of existing blocks. Blocks cannot overlap.
3.  Size Limit: The final machine must not exceed dimensions of 17 (Length, Z), 17 (Width, X), 9.5 (Height, Y).
4.  Functionality: Ensure functional blocks are oriented correctly.
5.  Ground Interaction: The ground automatically conforms to the machine's lowest block. Account for potential collisions between the machine and the ground throughout operation.
6.  Gravity: Every block is subject to gravity; the greater a block's mass, the stronger its downward force. Consider this in your design when the machine is in operation.
7.  Physical rules: Classical mechanical laws such as conservation of momentum are applied.

II. Block Data:
Notes:
You can only use blocks from this list. A block's default orientation is Z+.
1. Attachable face:
   a. `id`: The i-th attachable_face of this block.
   b. `pos`: Coordinates relative to the building center(which is the attachable_face of the parent block) of this block.
   c. `orientation`: Orientation relative to the building center of this block.
2. Tags:
   a. `Non-static`: Block can generate force or movement.
   b. `Non-stable`: Connection to parent is not rigid (e.g., hinges, boulders).
   c. `Linear`: Do not collide with other blocks, but will occupy two attachable_faces.
3. Special Blocks:
   a. Boulder (id 36): Does not physically connect to other blocks.
   b. Spring (id 9): A special block that pulls its two connection points together.

Detailed Infos:
<Block Infos without explanations>

III. JSON Output Format:
1. type: block's type_id
2. id: this is i-th block
3. parent: parent block's id
4. face_id: parent block's constructible_point id
5. Standard Block: `{"type": <int>, "id": <int>, "parent": <int>, "face_id": <int>}`
6. special block (id: 9): `{"type": 9, "id": <int>, "parent_a": <int>, "face_id_a": <int>, "parent_b": <int>, "face_id_b": <int>}`

IV. Final Response Format:
Your response must contain only these two parts:
1. `Chain of thoughts:` 
    a. You need to think step by step, analyse each block's usage, and where to place them. Put your cot in <cot></cot>
    b. `Construction Idea:` A brief explanation of your design, remember to consider necessary block types, note them in ```necessary_blocks [type_1,type_2 ...]```, no more than 300 words.
2.  `JSON:` The complete JSON code inside a ```json ... ``` block. Here is an example: 
```json
    [
        {"type":"0","id":0,"parent":-1,"face_id":-1},
        {"type": <int>, "id": <int>, "parent": <int>, "face_id": <int>},
        ...
    ]
```
\end{lstlisting}


\newpage
\section{Multi-Agent Prompts}
\subsection{Shared Prompts}
\subsubsection{Game Introduction With 3D Knowledge}
\begin{lstlisting}
1. Coordinate System: The game uses a left-handed coordinate system, with the Y-axis pointing upwards. In global coordinates, z+ is forward and x+ is to the right.
2. Construction: New blocks must be connected to the "attachable faces" of existing blocks. The default orientation of blocks is z+.
3. Block Types:
    a. Regular Blocks: Have fixed dimensions and multiple attachable faces.
    b. special blocks (ID 7, 9): Connect two attachable faces, do not collide with other blocks, but will occupy the connection points.
4. Size Limitations: The mechanical dimensions must not exceed Length (z) 17, Width (x) 17  Height (y) 9.5.
\end{lstlisting}

\subsubsection{Machine 3D JSON Format}
\begin{lstlisting}
```json
[
    {"type":"0","id":0,"parent":-1,"face_id":-1},
    {"type":"Block Type ID","id":"Block Order ID","parent":"Parent Block ID","face_id":"Attachable Face ID in Parent Block"},
    ...
]
```
If it is a special block (Type ID is 7 or 9, other blocks are not special blocks), it will be:
```json
{
    "type":"Block Type ID",
    "id":"Block Order ID",
    "parent_a":"Parent A Block Order ID",
    "face_id_a":"Attachable Face ID in Parent A Block",
    "parent_b":"Parent B Block Order ID",
    "face_id_b":"Attachable Face ID in Parent B Block"
}
```
\end{lstlisting}

\subsubsection{Build Guidance}
\begin{lstlisting}
Your task is to: Add new blocks based on the initial machine JSON and construction requirements provided by the user, without deleting the initial structure, and output the final complete JSON.

User Input Format:
1.  Building Objective: Describe the structure and function to be built, and provide a list of recommended block IDs.
2.  Initial JSON: The structural data of the existing machine.

Core Building Rules:
1.  Block Usage: You can only select from the list of recommended block IDs provided by the user. You may remove certain recommended block IDs due to "inapplicability" or "better alternatives," but you cannot add new IDs. If any are removed, the reason must be stated.
2.  Collision Prevention: You must accurately calculate the coordinates and orientation of new blocks based on the orientation and position of the parent block to ensure that the new block does not overlap with the existing structure.
3.  Coordinate System and Orientation: The initial orientation of all blocks is Z+. The final orientation of new blocks must be transformed based on the parent block's orientation and the relative direction of the building point, according to the following rules:
       Oriented z+: Front z+, Back z-, Left x-, Right x+, Up y+, Down y-
       Oriented z-: Front z-, Back z+, Left x+, Right x-, Up y+, Down y-
       Oriented x-: Front x-, Back x+, Left z-, Right z+, Up y+, Down y-
       Oriented x+: Front x+, Back x-, Left z+, Right z-, Up y+, Down y-
       Oriented y+: Front y+, Back y-, Left x-, Right x+, Up z-, Down z+
       Oriented y-: Front y-, Back y+, Left x-, Right x+, Up z+, Down z-

Your Output Format:
1.  Building Plan:
    `Generated [structure summary] to achieve [function]. Finally used blocks [ID1, ID2,...]. Removed [ID3,...] because [reason for removal].`
2.  Final JSON:
    ```json
    [
      // The complete JSON including both the initial structure and the new blocks
    ]
    ```    
\end{lstlisting}

\subsubsection{Meta Designer System Prompt}
\begin{lstlisting}
You are a mechanical designer, and your task is to design a machine in the game Besiege based on the user's requirements. Please gain a general understanding of the game based on the following information:

I. Game Introduction:
1. Besiege is a physics-based construction game developed using Unity. Players need to build various machines to complete different tasks.
2. Besiege only contains basic mechanics and physical laws, such as mass, friction, and collision.
3. Blocks are used to build machines. Each block has its unique functions, advantages, and disadvantages. 

II. Block Introduction:
1. Blocks are mainly divided into five major categories: Basic Blocks, Mobility Blocks, Mechanical Blocks, Weapon Blocks, and Armor Blocks.
- Basic Blocks are the fundamental components of many machines - structural blocks and some basic moving parts.
- Mobility Blocks are primarily designed for movement functions - powered and unpowered wheels, steering blocks, and gears.
- Mechanical Blocks provide various useful auxiliary functions - joints, suspension devices, winches, grabbers, etc.
- Weapon Blocks offer various types of violent output at different ranges - swords and saws for close combat, and cannons and rockets for long-range.
- Armor Blocks can protect the machine from damage or provide useful shapes for carrying other blocks - armor plates and wooden panels, as well as half-pipes and brackets.

2. Here is a detailed introduction to the properties and functions of each block:
| Name | Category | Type ID | Function |
|------|----------|---------|----------|
| Starting Block | Basic | 0 | The Starting Block is the root block of the machine; it is placed at the starting position by default, cannot be moved, cannot be deleted, and only one can exist at a time. |
| Small Wooden Block | Basic | 15 | A basic structural block, cube-shaped, to which other blocks can be attached from any side, making it particularly suitable for constructing the basic framework of machines. |
| Wooden Block | Basic | 1 | A basic mechanical block, twice the length of a Small Wooden Block. |
| Wooden Rod | Basic | 41 | A basic mechanical block, twice the length of a Small Wooden Block, with the same weight as a Wooden Block, but very fragile. |
| Log | Basic | 63 | A basic mechanical block, more robust, three times the length of a Small Wooden Block. |
| Brace | Basic | 7 | A non-placeable block used for reinforcement, built by "attaching" to other blocks, with no collision volume. It is often used to increase the stability of static structures and is not suitable for any dynamic structures. |
| Steering Hinge | Mobility | 28 | The Steering Hinge can rotate blocks along an axis perpendicular to the placement axis. This block can rotate child blocks to a 180-degree direction to the left or right, commonly used for vehicle steering. |
| Steering Block | Mobility | 13 | The Steering Block can rotate blocks along its placement axis, similar to the rotating part of a helicopter's rotor. |
| Powered Wheel | Mobility | 2 | Similar to a car wheel, it can drive itself but cannot turn independently. It is a mechanical device used for moving objects on the ground. |
| Unpowered Wheel | Mobility | 40 | A wheel that does not rotate without external force, otherwise similar to a Powered Wheel. |
| Powered Large Wheel | Mobility | 46 | Similar to a Powered Wheel, but with a radius and thickness twice that of a Powered Wheel. |
| Unpowered Large Wheel | Mobility |  | A wheel that does not rotate without external force, otherwise similar to a Powered Large Wheel. |
| Small Wheel | Mobility | 50 | It works almost the same as a caster wheel (like a shopping cart wheel), unpowered. |
| Universal Joint | Mechanical | 19 | A block that can freely rotate around its placement axis, similar to a Steering Block but without power. |
| Hinge | Mechanical | 5 | Similar to a Steering Hinge, but without power. |
| Ball Joint | Mechanical | 44 | Can swing 360 degrees along the axis perpendicular to the placement axis, but without power. |
| Axle Connector | Mechanical | 76 | Similar to a Ball Joint. |
| Rotating Block | Mechanical | 22 | Powered, it can rotate clockwise or counterclockwise along the axis perpendicular to the placement axis. |
| Suspension | Mechanical | 16 | Shaped like a wooden block, it can buffer forces from all directions. |
| Grabber | Mechanical | 27 | It will grab and hold onto any object it comes into contact with. |
| Spring | Mechanical | 9 | A special block that attaches to two other blocks and can quickly pull them together. Its pulling force is almost entirely dependent on its length. |
| Boulder | Weapon | 36 | A stone that does not directly connect to other blocks even when built on them. It can be used as a projectile weapon and is also commonly used as a target in transportation tests. |
| Elastic Pad | Armor | 87 | Increases the elasticity of the contact surface, providing an effect of rebounding and increasing kinetic energy. |
| Container | Armor | 30 | Can hold child blocks like a bowl, mainly used to carry blocks that cannot be directly connected to the machine. The container has some anti-slip capability, and only one block (the target to be carried) can be placed inside. No other blocks can be added. |
| Roller Wheel | Locomotion | 86 | Similar to the small wheel, but shorter (0.8m). |
|  Grip Pad | Armour | 49 | Block with the highest friction. |
|  Ballast | Flight | 35 | A heavy cubic block used as a counterweight. |

III. Mechanical Design Requirements:
1. When designing the machine, you should adopt a "layered design" approach. Break down the user's requirements into the functions that the machine needs to achieve, and list the functional points.
2. For each functional point, design a structure that can meet the function. A structure can be understood as a "group of blocks," and several structures combined form the machine.
3. For each structure, determine the types of blocks to be used.
4. Determine the construction order of the structures to make the machine-building process layered. List which structure is the foundation and which is the upper-layer structure, and establish the construction sequence chain.

IV. Output Format Requirements:
```json
{
    "definition": "Construct a machine that can fulfill the user's requirements",
    "function_points": ["Function Point 1", "Function Point 2", "Function Point 3"],
    "design_structure": [
        {
            "function_name": "Structure 1 Name",
            "description": "Description of Structure 1",
            "related_function_points": ["Function Point 1", "Function Point 2"]
        },
        {
            "function_name": "Structure 2 Name",
            "description": "Description of Structure 2",
            "related_function_points": ["Function Point 3"]
        }
    ],
    "build_order": ["Structure 2 Name", "Structure 1 Name"],
    "machine_structure": {
        "Structure 1 Name": {
            "block_id": [ID1, ID2, ID3...],
            "guidance": "Guidance here"
        },
        "Structure 2 Name": {
            "block_id": [ID4, ID5, ID6...],
            "guidance": "Guidance here"
        }
    }
}
```
V. Note:
1. You must design the machine based on the game introduction and block introduction, and you cannot use blocks that do not exist in the game.
2. Strictly follow the output format requirements. Do not output any content other than what is required by the output format.
3. For the design of structures, aim for simplicity and use the minimum number of structures to complete all functions. Check if there are existing structures that can be used before designing new ones.
4. When selecting blocks for a structure, limit the types of blocks to no more than three, and preferably use only one type. Focus solely on meeting the functional points with the bare minimum requirements, and do not attempt to fulfill demands beyond the functional points.

I will provide the user input below. Please generate a mechanical overview in JSON format based on the user's description.
\end{lstlisting}


\subsection{Designer System And User Prompt}
\begin{lstlisting}
<system>
You are a mechanical builder in the game "Besiege." 
Your task is to add new blocks to an existing machine structure according to user requests and finally output the complete machine JSON data.
I. Game Introduction:
<Game Introduction With 3D Knowledge>
II. Introduction to Blocks:
<Block Infos>
III. Introduction to JSON Format:
<Machine 3D JSON Format>
IV. Construction Guidance:
<Build Guidance>
V. Output Format Requirements:
Based on the existing structure, a [structural summary] was generated as [functional implementation]. 
Ultimately, the block types [ID1, ID2, ...] were decided upon, while [ID3, ...] were removed due to [reason for removal].
JSON:
```json
...
```
VI. Note:
Building Principles
1. Correct Orientation: Ensure that functional blocks such as wheels and hinges are oriented correctly to achieve the intended function.
2.  Efficiency First:
    a. Complete the design goal with the fewest blocks possible.
    b. The ground will automatically adapt to the lowest point of the mechanism.

Output Requirements
1.  Strict Structure: Your response must only contain two parts: Construction Plan and Final JSON. Prohibit any additional greetings, explanations, or comments.
2.  Pure JSON: The complete JSON code block must be placed within ```json ... ```. Prohibit modifying the initial structure. Prohibit modifying the `scale` property of any block. Prohibit adding comments or non-existent properties in the JSON.

Next, I will provide user input. Please generate a JSON based on the description.
<user>
<designer_output["design_structure"][i]["description"]>+<Output Machine Json>
\end{lstlisting}

\subsection{Inspector System And User Prompt}
\begin{lstlisting}
<system>
I'll provide you with a mission in the game Besiege, 
along with the machine designed for it in JSON format and its 3D information. 
Please identify and summarize the unreasonable parts of the machine design. 
Here's the introduction to the game and construction knowledge.
I. Game Introduction:
<Game Introduction With 3D Knowledge>
II. Introduction to Blocks:
<Block Infos>
III. Introduction to JSON and 3D Information:
<Machine 3D JSON Format>
IV. Introduction to Output Format:
<Three-Dimensional Perception of the World>
{
    "Coordinate System": "Left-Handed System or Right-Handed System",
    "Up": "x or y or z",
    "Right": "x or y or z",
    "Forward": "x or y or z",
    "Frontmost Block": {"id": [Center Point Coordinates]},
    "Rearmost Block": {"id": [Center Point Coordinates]},
    "Leftmost Block": {"id": [Center Point Coordinates]},
    "Rightmost Block": {"id": [Center Point Coordinates]},
    "Topmost Block": {"id": [Center Point Coordinates]},
    "Lowest Block": {"id": [Center Point Coordinates]},
}
</Three-Dimensional Perception of the World>

<Question Answer>
Write the answer to the user's question here
</Question Answer>

<Summary of Design Defects>
1. Problem description, involving blocks: [id_list], belonging to structure: "Structure Name"
...
</Summary of Design Defects>
V. Notes:
1. Please do not output any irrelevant information and directly answer the user's question.
2. The id of the block must be enclosed in "[]". Additionally, do not use "[]" for any other numbers; consider using "()" instead.

Below, I will provide you with JSON and 3D information. Please answer the user's question based on this information.

<user>
Task Introduction
{designer_output["definition"]}
JSON Information
<Output Machine Json>
3D Information
<Output Machine Json 3D Info>
Mechanical Structure Information
<Machine Tree With Designer Layer Arrangement>
Initial State of the Machine
The machine is initially placed on the ground, facing in the z+ direction, with the target direction being z+.

Questions:
1. Output the position and orientation of all dynamic blocks, and analyze:
    a. The impact of dynamic blocks on the machine
    b. The direction of force provided by dynamic blocks
    c. The impact on sub-blocks and the direction of force on the machine

2. Output static blocks other than basic structural blocks, and analyze the rationality of their orientation and position.

3. Balance Check (self-gravity)
    a. The center of gravity of the machine (find the block closest to the center of gravity)
    b. Whether parts of the machine will sink due to gravity
4. Comprehensive Analysis
    a. Summarize the direction of all forces to analyze the movement of the machine
    b. Identify logically unreasonable blocks, output their hierarchical structure and reasons for unreasonableness
\end{lstlisting}

\subsection{Refiner System Prompt}
\begin{lstlisting}
I will give you a task in the game Besiege, as well as the 3D information of the machine designed to complete this task. There are some unreasonable aspects in the design of this machine, and I would like you to modify these parts:
I. Game Introduction:
<Game Introduction With 3D Knowledge>
II. Block Introduction:
<Block Infos>
III. Input Introduction:
1. Task & Context
Task Objective:
   <designer_output["user_input"]>
Preceding Information (if any):
   Defect Report:
       <quizzer_output>
   Modification History:
       <modify_history>
   Environmental Feedback:
       <environment_feedback>
       
2. Machine Data
<Machine 3D JSON Format>

IV. Modification Method Introduction:
Please follow the steps below to analyze and modify the machine structure.
Step 1: Analyze & Plan
1.  Diagnose the Current Machine: Analyze its power, balance, structure, and overall movement to identify design flaws or areas for optimization.
2.  Devise a Modification Plan: Based on your diagnosis, decide which blocks to move, remove, or add.
3.  Evaluate Modification Impact: When planning, you must consider the impact of your changes.
4.  Briefly Describe Modifications: Before generating commands, describe your modification plan in a sentence or two using natural language.

Step 2: Execute Modification Operations
Use the following three command formats for modifications. Output only the operation commands, with each command on a new line.

1.  Add Block (Add)
   Format: Non-Linear: Add [block type ID] to [id] in [attachable_face_id]
            Linear: Add [block type ID] to [id_a] in [attachable_face_id_a] to [id_b] in [attachable_face_id_b]
   Rules:
       Can only be added to original blocks. You cannot add new blocks onto other newly added blocks.
       special blocks require specifying two connection points.

2.  Remove Block (Remove)
   Format: Remove [id]
   Rules:
       Can only remove original blocks.
       Cannot remove a block that has child blocks.

3.  Move Block (Move)
   Move [id] to [new_parent_id] in [new_attachable_face_id]
   Rules:
       Moves the target block and all its child blocks as a single unit.
       The new parent's `id` must be smaller than the `id` of the block being moved.
       special blocks cannot be moved.
       The move must change the block's original position.
       
<Build Guidance: Coordinate System and Orientation>

V. Output Format Introduction:
<Thought Process>
</Thought Process>
<Modification Description>
</Modification Description>
<Simulation Prediction After Modification>
</Simulation Prediction After Modification>
<Required Feedback>
</Required Feedback>
<Modification Steps>
</Modification Steps>
VI. Note:
1. Output Scope: Only output content related to the modification method and the required output format.
2. Ground Definition: The ground is always located beneath the machine, in contact with the block that has the lowest y-coordinate.
3. Task Adherence: Pay close attention to the task requirements. Do not delete important blocks by mistake.
4. Parenting Constraint: A block that has been deleted cannot be used as a parent for new or moved blocks within the same set of operations.
5. Format Integrity: Ensure the output format is complete. You must retain the `<Thought Process></Thought Process>` and `<Modification Description></Modification Description>` tags.
6. Content Separation: Do not include verification results within the `<Modification Description>` block.
7. Preserve 'Success' Steps: Retain any steps marked as "Success" and do not adjust their order.
8. Prioritize 'Error' Steps: Focus your efforts on fixing the steps marked as "Error". Do not modify steps marked as "Unverified" prematurely.
9. Error History: I will provide the modification history for the "Error" steps. Use this information to avoid repeating invalid operations.

Below, I will provide you with the JSON and 3D information. Please modify the machine accordingly.
\end{lstlisting}

\subsection{Environment Querier System Prompt}
\begin{lstlisting}
I will give you a task in the game Besiege, as well as the information on the machine designed to complete this task. 
The machine has finished the task simulation and returned some environmental feedback to describe its performance.
Please analyze the issues with the machine based on the feedback and request more feedback if needed.
I. Game Introduction:
<Game Introduction With 3D Knowledge>
II. Block Introduction:
<Block Infos>
III. Input Introduction:
<Machine 3D JSON Format>
IV. Query Introduction:
A. Environmental Feedback Request:
After conducting the environmental simulation of your modified machinery, 
The system will provide essential data for the most critical components (such as position, rotation, and velocity). 
Based on the performance of the essential data, you need to determine what problems the machinery may have encountered and request feedback on the key blocks that may have issues.
You can also request those blocks that may significantly impact the machinery's functionality and check whether their performance meets expectations.
The format for the environmental feedback request is as follows. 
Please adhere strictly to this format and avoid including any extraneous information:
<Required Feedback>
[
    {
        "id": int,
        "duration": [float, float],
        "properties": ["position", "rotation", "velocity", "length"],
    },
    ...
]
</Required Feedback>
Both "id" and "duration" are mandatory fields. 
Note that the game runs for a total of 5 seconds, game state samples per 0.2s; do not exceed this duration. 
You can freely select the attributes in "properties," but avoid including irrelevant information. 
The "length" attribute is only applicable to linear components.
V. Output Format Introduction:
<Thought Process>
</Thought Process>
<Required Feedback>
</Required Feedback>
VI. Note:
1. Please do not output any irrelevant information.
2. The ground will always correctly appear beneath the machine, making normal contact with the block that has the lowest y-axis coordinate.
3. All blocks are affected by gravity. If a block with a large self-weight is built on certain non-powered non-static blocks, the non-static blocks may rotate due to the gravitational force of the sub-blocks.
4. Similarly, the power generated by powered blocks also needs to counteract the gravitational force of the sub-blocks.
5. Please adhere to the output format and avoid adding any irrelevant information in the JSON.

Below, I will provide you with the JSON and 3D information, as well as the environmental feedback. Please request more feedback as needed.
\end{lstlisting}

\subsection{Block Informations}
\begin{lstlisting}
Explanations:

This is a concise list of the blocks you can use for this construction. Please read and follow the rules carefully.
I. Block Information Format
Each block's information follows the dict format. Attributes that a block does not have will not appear in the keys.
1. Characteristic Tags:
   a. Non-static: The block can actively generate force or movement.
   b. Non-stable: The connection between the block and its parent block is non-rigid, allowing for movement or rotation.
   c. Linear: The block is used to connect two existing points rather than being attached to a single point.
   If there are no tags, it is a regular static and stable block.
2. Attachable Faces:
   The key is in the format of attachable face ID, coordinates (relative coordinates), and orientation (relative orientation).
II. Key Special Rules
1. Powered Wheel Rule (applicable to all powered wheels): The direction of power provided by the wheel is not the same as its orientation.
   - Forward (Z+ direction): The wheel should face sideways (X+ or X-).
   - Left turn (power pushes towards X-): The wheel should face forward (Z+).
   - Right turn (power pushes towards X+): The wheel should face backward (Z-).
   - When the wheel faces up or down (Y+ or Y-), it does not provide power.
2. Special Blocks (Brace, Spring):
   - They do not have their own connection points but connect to two other blocks.
   - Brace: Used to reinforce static structures with no collision volume.
   - Spring: Generates a contracting pull force along the line connecting its two ends when stretched.
3. Non-connecting Blocks (bombs, boulders):
   - These blocks are placed at the specified location but do not form a physical connection with any other block. Containers are usually needed to hold them.

Detailed Infos:
[
    {
        "Name": "Starting Block",
        "Description": "The root block of the mechanism. It cannot be placed or deleted, and only one can exist at a time. Its initial position is fixed, and its initial orientation is z+.",
        "Type ID": 0,
        "Size": [1, 1, 1],
        "Attachable Faces Properties": [
            {"ID": 0, "Coordinates": [0, 0, 0.5], "Orientation": "Front"},
            {"ID": 1, "Coordinates": [0, 0, -0.5], "Orientation": "Back"},
            {"ID": 2, "Coordinates": [-0.5, 0, 0], "Orientation": "Left"},
            {"ID": 3, "Coordinates": [0.5, 0, 0], "Orientation": "Right"},
            {"ID": 4, "Coordinates": [0, 0.5, 0], "Orientation": "Up"},
            {"ID": 5, "Coordinates": [0, -0.5, 0], "Orientation": "Down"}
        ],
        "Mass": 0.25
    },
    {
        "Name": "Small Wooden Block",
        "Description": "A basic construction block, cubic in shape.",
        "Type ID": 15,
        "Size": [1, 1, 1],
        "Attachable Faces Properties": [
            {"ID": 0, "Coordinates": [0, 0, 1], "Orientation": "Front"},
            {"ID": 1, "Coordinates": [-0.5, 0, 0.5], "Orientation": "Left"},
            {"ID": 2, "Coordinates": [0.5, 0, 0.5], "Orientation": "Right"},
            {"ID": 3, "Coordinates": [0, 0.5, 0.5], "Orientation": "Up"},
            {"ID": 4, "Coordinates": [0, -0.5, 0.5], "Orientation": "Down"}
        ],
        "Mass": 0.3
    },
    {
        "Name": "Wooden Block",
        "Description": "A basic construction block.",
        "Type ID": 1,
        "Size": [1, 1, 2],
        "Attachable Faces Properties": [
            {"ID": 0, "Coordinates": [0, 0, 2], "Orientation": "Front"},
            {"ID": 1, "Coordinates": [-0.5, 0, 0.5], "Orientation": "Left"},
            {"ID": 2, "Coordinates": [-0.5, 0, 1.5], "Orientation": "Left"},
            {"ID": 3, "Coordinates": [0.5, 0, 0.5], "Orientation": "Right"},
            {"ID": 4, "Coordinates": [0.5, 0, 1.5], "Orientation": "Right"},
            {"ID": 5, "Coordinates": [0, 0.5, 0.5], "Orientation": "Up"},
            {"ID": 6, "Coordinates": [0, 0.5, 1.5], "Orientation": "Up"},
            {"ID": 7, "Coordinates": [0, -0.5, 0.5], "Orientation": "Down"},
            {"ID": 8, "Coordinates": [0, -0.5, 1.5], "Orientation": "Down"}
        ],
        "Mass": 0.5
    },
    {
        "Name": "Wooden Rod",
        "Description": "A basic construction block, slender and fragile.",
        "Type ID": 41,
        "Size": [1, 1, 2],
        "Attachable Faces Properties": [
            {"ID": 0, "Coordinates": [0, 0, 2], "Orientation": "Front"},
            {"ID": 1, "Coordinates": [-0.5, 0, 0.5], "Orientation": "Left"},
            {"ID": 2, "Coordinates": [-0.5, 0, 1.5], "Orientation": "Left"},
            {"ID": 3, "Coordinates": [0.5, 0, 0.5], "Orientation": "Right"},
            {"ID": 4, "Coordinates": [0.5, 0, 1.5], "Orientation": "Right"},
            {"ID": 5, "Coordinates": [0, 0.5, 0.5], "Orientation": "Up"},
            {"ID": 6, "Coordinates": [0, 0.5, 1.5], "Orientation": "Up"},
            {"ID": 7, "Coordinates": [0, -0.5, 0.5], "Orientation": "Down"},
            {"ID": 8, "Coordinates": [0, -0.5, 1.5], "Orientation": "Down"}
        ],
        "Mass": 0.5
    },
    {
        "Name": "Log",
        "Description": "A basic construction block.",
        "Type ID": 63,
        "Size": [1, 1, 3],
        "Attachable Faces Properties": [
            {"ID": 0, "Coordinates": [0, 0, 3], "Orientation": "Front"},
            {"ID": 1, "Coordinates": [-0.5, 0, 0.5], "Orientation": "Left"},
            {"ID": 2, "Coordinates": [-0.5, 0, 1.5], "Orientation": "Left"},
            {"ID": 3, "Coordinates": [-0.5, 0, 2.5], "Orientation": "Left"},
            {"ID": 4, "Coordinates": [0.5, 0, 0.5], "Orientation": "Right"},
            {"ID": 5, "Coordinates": [0.5, 0, 1.5], "Orientation": "Right"},
            {"ID": 6, "Coordinates": [0.5, 0, 2.5], "Orientation": "Right"},
            {"ID": 7, "Coordinates": [0, 0.5, 0.5], "Orientation": "Up"},
            {"ID": 8, "Coordinates": [0, 0.5, 1.5], "Orientation": "Up"},
            {"ID": 9, "Coordinates": [0, 0.5, 2.5], "Orientation": "Up"},
            {"ID": 10, "Coordinates": [0, -0.5, 0.5], "Orientation": "Down"},
            {"ID": 11, "Coordinates": [0, -0.5, 1.5], "Orientation": "Down"},
            {"ID": 12, "Coordinates": [0, -0.5, 2.5], "Orientation": "Down"}
        ],
        "Mass": 1
    },
    {
        "Name": "Steering Hinge",
        "Description": "Powered, used to control the rotation of sub-blocks. It can swing left and right along the axis perpendicular to the placement axis.",
        "Type ID": 28,
        "Size": [1, 1, 1],
        "Attachable Faces Properties": [
            {"ID": 0, "Coordinates": [0, 0, 1], "Orientation": "Front"}
        ],
        "Special Attributes": {
            "Swing Direction": ["Left", "Right"],
            "Angle": [-90, 90],
            "NonStatic":"True",
            "NonStable":"True"
        },
        "Mass": 1
    },
    {
        "Name": "Steering Block",
        "Description": "Powered, used to control the rotation of sub-blocks. It can rotate clockwise or counterclockwise along the placement axis.",
        "Type ID": 13,
        "Size": [1, 1, 1],
        "Attachable Faces Properties": [
            {"ID": 0, "Coordinates": [0, 0, 1], "Orientation": "Front"},
            {"ID": 1, "Coordinates": [-0.5, 0, 0.5], "Orientation": "Left"},
            {"ID": 2, "Coordinates": [0.5, 0, 0.5], "Orientation": "Right"},
            {"ID": 3, "Coordinates": [0, 0.5, 0.5], "Orientation": "Up"},
            {"ID": 4, "Coordinates": [0, -0.5, 0.5], "Orientation": "Down"}
        ],
        "Special Attributes": {
            "Rotation Axis": "Front",
            "NonStatic":"True",
            "NonStable":"True"
        },
        "Mass": 1
    },
    {
        "Name": "Powered Wheel",
        "Description": "Powered, a mechanical device used to move objects on the ground.",
        "Type ID": 2,
        "Size": [2, 2, 0.5],
        "Attachable Faces Properties": [
            {"ID": 0, "Coordinates": [0, 0, 0.5], "Orientation": "Front"}
        ],
        "Special Attributes": {
            "Rotation Axis": "Front",
            "PoweredWheel":"True",
            "NonStatic":"True",
            "NonStable":"True"
        },
        "Mass": 1
    },
    {
        "Name": "Unpowered Wheel",
        "Description": "A wheel that does not rotate without external force, similar to the powered wheel.",
        "Type ID": 40,
        "Size": [2, 2, 0.5],
        "Attachable Faces Properties": [
            {"ID": 0, "Coordinates": [0, 0, 0.5], "Orientation": "Front"}
        ],
        "Special Attributes": {
            "Rotation Axis": "Front",
            "NonStable":"True"
        },
        "Mass": 1
    },
    {
        "Name": "Large Powered Wheel",
        "Description": "Similar to the powered wheel, but larger.",
        "Type ID": 46,
        "Size": [3, 3, 1],
        "Attachable Faces Properties": [
            {"ID": 0, "Coordinates": [0, 0, 1], "Orientation": "Front"},
            {"ID": 1, "Coordinates": [-1.5, 0, 1], "Orientation": "Front"},
            {"ID": 2, "Coordinates": [1.5, 0, 1], "Orientation": "Front"},
            {"ID": 3, "Coordinates": [0, 1.5, 1], "Orientation": "Front"},
            {"ID": 4, "Coordinates": [0, -1.5, 1], "Orientation": "Front"},
            {"ID": 5, "Coordinates": [-1.5, 0, 0.5], "Orientation": "Left"},
            {"ID": 6, "Coordinates": [1.5, 0, 0.5], "Orientation": "Right"},
            {"ID": 7, "Coordinates": [0, 1.5, 0.5], "Orientation": "Up"},
            {"ID": 8, "Coordinates": [0, -1.5, 0.5], "Orientation": "Down"}
        ],
        "Special Attributes": {
            "Rotation Axis": "Front",
            "PoweredWheel":"True",
            "NonStatic":"True",
            "NonStable":"True"
        },
        "Mass": 1
    },
    {
        "Name": "Large Unpowered Wheel",
        "Description": "Similar to the unpowered wheel, but larger.",
        "Type ID": 60,
        "Size": [3, 3, 1],
        "Attachable Faces Properties": [
            {"ID": 0, "Coordinates": [0, 0, 1], "Orientation": "Front"},
            {"ID": 1, "Coordinates": [-1.5, 0, 1], "Orientation": "Front"},
            {"ID": 2, "Coordinates": [1.5, 0, 1], "Orientation": "Front"},
            {"ID": 3, "Coordinates": [0, 1.5, 1], "Orientation": "Front"},
            {"ID": 4, "Coordinates": [0, -1.5, 1], "Orientation": "Front"},
            {"ID": 5, "Coordinates": [-1.5, 0, 0.5], "Orientation": "Left"},
            {"ID": 6, "Coordinates": [1.5, 0, 0.5], "Orientation": "Right"},
            {"ID": 7, "Coordinates": [0, 1.5, 0.5], "Orientation": "Up"},
            {"ID": 8, "Coordinates": [0, -1.5, 0.5], "Orientation": "Down"}
        ],
        "Special Attributes": {
            "Rotation Axis": "Front",
            "NonStable":"True"
        },
        "Mass": 1
    },
    {
        "Name": "Small Wheel",
        "Description": "It works almost the same as a caster wheel (e.g., shopping cart wheel), but it is not powered.",
        "Type ID": 50,
        "Size": [0.5, 1, 1.5],
        "Special Attributes": {"NonStable":"True"},
        "Mass": 0.5
    },
    {
        "Name": "Roller Wheel",
        "Description": "Same as the small wheel.",
        "Type ID": 86,
        "Size": [1, 1, 1],
        "Special Attributes": {
            "NonStable":"True"
        },
        "Mass": 0.5
    },
    {
        "Name": "Universal Joint",
        "Description": "A block that can freely rotate around its placement axis, but it is not powered.",
        "Type ID": 19,
        "Size": [1, 1, 1],
        "Attachable Faces Properties": [
            {"ID": 0, "Coordinates": [0, 0, 1], "Orientation": "Front"},
            {"ID": 1, "Coordinates": [-0.5, 0, 0.5], "Orientation": "Left"},
            {"ID": 2, "Coordinates": [0.5, 0, 0.5], "Orientation": "Right"},
            {"ID": 3, "Coordinates": [0, 0.5, 0.5], "Orientation": "Up"},
            {"ID": 4, "Coordinates": [0, -0.5, 0.5], "Orientation": "Down"}
        ],
        "Special Attributes": {
            "Rotation Axis": "Front",
            "NonStable":"True"
        },
        "Mass": 0.5
    },
    {
        "Name": "Hinge",
        "Description": "It can swing up and down along the axis perpendicular to the placement axis, but it is not powered.",
        "Type ID": 5,
        "Size": [1, 1, 1],
        "Attachable Faces Properties": [
            {"ID": 0, "Coordinates": [0, 0, 1], "Orientation": "Front"},
            {"ID": 1, "Coordinates": [-0.5, 0, 0.5], "Orientation": "Left"},
            {"ID": 2, "Coordinates": [0.5, 0, 0.5], "Orientation": "Right"},
            {"ID": 3, "Coordinates": [0, 0.5, 0.5], "Orientation": "Up"},
            {"ID": 4, "Coordinates": [0, -0.5, 0.5], "Orientation": "Down"}
        ],
        "Special Attributes": {
            "Swing Direction": ["Up", "Down"],
            "Angle": [-90, 90],
            "NonStable":"True"
        },
        "Mass": 0.5
    },
    {
        "Name": "Ball Joint",
        "Description": "It can swing freely in all directions, but it is not powered.",
        "Type ID": 44,
        "Size": [1, 1, 1],
        "Attachable Faces Properties": [
            {"ID": 0, "Coordinates": [0, 0, 1], "Orientation": "Front"},
            {"ID": 1, "Coordinates": [-0.5, 0, 0.5], "Orientation": "Left"},
            {"ID": 2, "Coordinates": [0.5, 0, 0.5], "Orientation": "Right"},
            {"ID": 3, "Coordinates": [0, 0.5, 0.5], "Orientation": "Up"},
            {"ID": 4, "Coordinates": [0, -0.5, 0.5], "Orientation": "Down"}
        ],
        "Special Attributes": {
            "Swing Range": "All directions outward from the build surface",
            "NonStable":"True"
        },
        "Mass": 0.5
    },
    {
        "Name": "Axle Connector",
        "Description": "Similar to a ball joint.",
        "Type ID": 76,
        "Size": [1, 1, 1],
        "Attachable Faces Properties": [
            {"ID": 0, "Coordinates": [0, 0, 1], "Orientation": "Front"}
        ],
        "Special Attributes": {
            "Swing Range": "All directions outward from the build surface",
            "NonStable":"True"
        },
        "Mass": 0.3
    },
    {
        "Name": "Rotating Block",
        "Description": "When powered, this motor-like block generates torque and rotates about its local y-axis. Blocks connected at attachable_face 1 or 4 rotate with it as part of a rigid assembly. The rotation block has its own mass and obeys classical mechanics: it applies torque to connected parts when powered, and it can also be moved, rotated, or stopped by external forces or torques, depending on constraints.",
        "Type ID": 22,
        "Size": [1, 1, 1],
        "Attachable Faces Properties": [
            {"ID": 0, "Coordinates": [0, 0, 1], "Orientation": "Front"},
            {"ID": 1, "Coordinates": [-0.5, 0, 0.5], "Orientation": "Left"},
            {"ID": 2, "Coordinates": [0.5, 0, 0.5], "Orientation": "Right"},
            {"ID": 3, "Coordinates": [0, 0.5, 0.5], "Orientation": "Up"},
            {"ID": 4, "Coordinates": [0, -0.5, 0.5], "Orientation": "Down"}
        ],
        "Special Attributes": {
            "Rotation Axis": "Front",
            "NonStatic":"True",
            "NonStable":"True"
        },
        "Mass": 1
    },
    {
        "Name": "Grabber",
        "Description": "If the build point is unoccupied, it will grab any object that comes into contact with the build point and hold it firmly.",
        "Type ID": 27,
        "Size": [1, 1, 1],
        "Attachable Faces Properties": [
            {"ID": 0, "Coordinates": [0, 0, 1], "Orientation": "Front"}
        ],
        "Special Attributes": {
            "Grip Direction": "Front",
            "NonStable":"True"
        },
        "Mass": 0.5
    },
    {
        "Name": "Boulder",
        "Description": "A rock that will not directly connect to other blocks even if built on them, high mass.",
        "Type ID": 36,
        "Size": [1.9, 1.9, 1.9],
        "Special Attributes": {
            "NonStable":"True"
        },
        "Mass": 5
    },
    {
        "Name": "Grip Pad",
        "Description": "The block with the highest friction.",
        "Type ID": 49,
        "Size": [0.8, 0.8, 0.5],
        "Mass": 0.3
    },
    {
        "Name": "Elastic Pad",
        "Description": "The block with the highest elasticity.",
        "Type ID": 87,
        "Size": [0.8, 0.8, 0.2],
        "Mass": 0.3
    },
    {
        "Name": "Container",
        "Description": "It has a railing around the building point. If oriented towards +y, it can hold sub-blocks like a bowl. It is mainly used to hold blocks that cannot directly connect to the mechanism, such as boulders and bombs. Do not place other blocks nearby to avoid overlap.",
        "Type ID": 30,
        "Size": [2.4, 3, 2.8],
        "Attachable Faces Properties": [
            {"ID": 0, "Coordinates": [0, 0, 1], "Orientation": "Front"}
        ],
        "Mass": 0.5
    },
    {
        "Name": "Suspension",
        "Description": "It primarily serves as a buffer and shock absorber. It is similar in shape to a wooden block, with all Attachable Faces Properties located at the far end of the block.",
        "Type ID": 16,
        "Size": [1, 1, 2],
        "Attachable Faces Properties": [
            {"ID": 0, "Coordinates": [0, 0, 2], "Orientation": "Front"},
            {"ID": 1, "Coordinates": [-0.5, 0, 1.5], "Orientation": "Left"},
            {"ID": 2, "Coordinates": [0.5, 0, 1.5], "Orientation": "Right"},
            {"ID": 3, "Coordinates": [0, 0.5, 1.5], "Orientation": "Up"},
            {"ID": 4, "Coordinates": [0, -0.5, 1.5], "Orientation": "Down"}
        ],
        "Mass": 0.5
    },
    {
        "Name": "Brace",
        "Description": "The brace can be used for reinforcement. Its construction principle is to 'attach' to other blocks. It has no collision volume. Since it is often used to stabilize static structures, it is not suitable for any dynamic structures.",
        "Type ID": 7,
        "Special Attributes": {
            "Linear": "True",
            "Anti Tension Direction": "Towards the center of the line segment between the two Attachable Faces Properties",
            "Anti-Compression Direction": "Outward from the center of the line segment between the two Attachable Faces Properties"
        },
        "Mass": 0.5
    },
    {
        "Name": "Spring",
        "Description": "A special block that attaches to two other blocks and can quickly pull the two ends together. Its tension force is almost entirely dependent on its length.",
        "Type ID": 9,
        "Special Attributes": {
            "Linear": "True",
            "NonStatic":"True",
            "Tension Direction": "Towards the center of the line segment between the two Attachable Faces Properties"
        },
        "Mass": 0.4
    },
    {
        "Name": "Ballast",
        "Description": "It serves as a counterweight, has a large mass, and is shaped like a cube.",
        "Type ID": 35,
        "Size": [1, 1, 1],
        "Attachable Faces Properties": [
            {"ID": 0, "Coordinates": [0, 0, 1], "Orientation": "Front"},
            {"ID": 1, "Coordinates": [-0.5, 0, 0.5], "Orientation": "Left"},
            {"ID": 2, "Coordinates": [0.5, 0, 0.5], "Orientation": "Right"},
            {"ID": 3, "Coordinates": [0, 0.5, 0.5], "Orientation": "Up"},
            {"ID": 4, "Coordinates": [0, -0.5, 0.5], "Orientation": "Down"}
        ],
        "Mass": 3
    }
]
\end{lstlisting}
